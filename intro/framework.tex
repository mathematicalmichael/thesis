\subsection{Terminology, notation, and the inverse problems}

To make comparisons more clear, we first introduce shared notation between the SIP and Bayesian inverse problems.
Denote by $\pspace$ the space of physically relevant parameters for the model.
Denote by $Q$ the (potentially vector-valued) QoI map from the parameter space, $\pspace$, to the data space defined by $\dspace:= Q(\pspace)$.

We provide a summary of the notation, definitions, problem-formulation, and assumptions that reoccur throughout this work.
For more details on the original sources and derivations,  we refer the interested reader to \cite{BES12, BE13, BET+14}.

Let $u$ be the solution to a model $\M(u, \param) = 0$, perhaps of a physical system such as the amount of contaminant in a subsurface.
Let $\param$ represent a parameter into such a model, e.g. the permeability of the medium in the subsurface through which a contaminant is spreading.
Such parameters are often uncertain and we begin the quantification of uncertainty by identifying the set of all physically plausible parameters denoted by $\pspace\subset\RR^\dimP$.
Since different choices of $\param \in \pspace$ often lead to different model solutions, we write $u\lam$ to make this dependence explicit.

However, we cannot in general observe the entire solution $u(\param)$.
Instead, we are often limited in our ability to observe data related to some quantities of interest (QoI), defined as functionals of $u\lam$ (e.g. the contaminant levels at a specific well at a particular time).
We let $\qoi$ denote the QoI map from the solution space of the model to the space of observable data.
Then, given $\param \in \pspace$, we obtain $u\lam$ and compute $\qoi(u\lam)$ to get the QoI datum predicted by the model.
Clearly, the QoI map depends on $\param$ through the dependency of $u$ on $\param$, so we write $\qlam$ to simplify our notation.
We assume this map is at least piecewise-differentiable.
The data space $\dspace \subset \RR^d$ is defined as the range of the QoI map $\qoi$, i.e.
\[
\dspace = \qoi(\pspace).
\]
In other words, we use $\dspace$ to denote the space of all physically plausible data for the QoI that the model can predict.


Let $\pborel$ and $\dborel$ denote (the Borel) $\sigma$-algebras on $\pspace$ and $\dspace$, respectively.
The map $\qoi$ between measurable spaces $(\pspace, \pborel)$ and $(\dspace, \dborel)$ is immediately measurable by the smoothness assumption.
Then, equipping $\pspace$ and $\dspace$ with (dominating) measures $\pmeas$ and $\dmeas$, respectively, is the final ingredient for defining probability density functions (pdfs) on the measure spaces $(\pspace, \pborel, \pmeas)$ and $(\dspace, \dborel, \dmeas)$.
In practice, $\pmeas$ and $\dmeas$ are often taken to be Lebesgue volume measures when $\pspace$ and $\dspace$ are finite-dimensional~\cite{BET+14, BJW18}.
In general, these measure allow for the description of probability measures as probability density functions defined by Radon-Nikodym derivatives.


\subsection{Problem Formulation and Solution}

We begin with defining the types of forward and inverse problems considered in this thesis.

\begin{defn}[Forward Problem]\label{defn:forward-problem}
  Given a probability measure $\PP_\pspace$ on $(\pspace, \pborel)$, and (at least piecewise-differentiable) QoI map $\qoi$, the \emph{forward problem} is to characterize a measure $\PP_\dspace$ on $(\dspace, \dborel)$ (recalling $\dspace = \qoi(\pspace)$) that satisfies
  \begin{equation}\label{eq:forward-problem}
    \PP_\dspace (E) = \PP_\pspace \left ( \qoi^{-1}(E) \right ), \; \forall \; E \in \dborel.
  \end{equation}
\end{defn}

\begin{defn}[Inverse Problem]\label{defn:inverse-problem}
  Given a probability measure $\observedP$ on $(\dspace, \dborel)$ that is absolutely continuous with respect to volume measure $\dmeas$, the \emph{inverse problem} is to determine a probability measure $\PP_\pspace$ on $(\pspace, \pborel)$, absolutely continuous with respect to $\pmeas$, satisfying

  \begin{equation}\label{eq:inverse-problem}
    \PP_\pspace (\qoi^{-1}(E)) = \int_{\qoi^{-1}(E)} \pp_\pspace \lam \, d\pmeas = \int_E \observed \Q \, d\dmeas = \observedP(E), \; \forall \; E \in \mathcal{B}_\dspace.
  \end{equation}

  \noindent Here,

  \begin{equation*}
    \pp_\pspace := \frac{d\PP_\pspace}{d\pmeas} \;\text{ and }\; \observed := \frac{d\observedP}{d\dmeas}
  \end{equation*}
  denote the Radon-Nikodym derivatives (i.e., pdfs) of $\updatedP$ and $\observedP$, respectively.
  Any probability measure $\PP_\pspace$ satisfying \eqref{eq:inverse-problem} is referred to as a \emph{consistent solution} to the inverse problem, and \eqref{eq:inverse-problem} is referred to as the \emph{consistency condition}.
\end{defn}

In measure-theoretic terms, $\PP_\pspace$ is a pull-back measure of $\observedP$.
From the perspective of a forward problem, we seek $\PP_\pspace$ such that its \emph{push-forward measure is equivalent to} $\observedP$.
In other words, the solution we seek to the inverse problem is constrained by a forward problem.
Below, we formalize some of the vocabulary involved in the formulation and solution of the stochastic inverse problem.

\begin{defn}[Observed Distribution]\label{defn:observed}
  The density $\observed$ in \eqref{eq:inverse-problem} represents the uncertainty in QoI data and is referred to as the \emph{observed distribution} (or density), and is the Radon-Nikodym derivative of the \emph{observed measure} $\observedP$ with respect to the volume measure $\dmeas$.
\end{defn}

The map $\qoi$ impacts the structure of the update since the underlying data space $\dspace$ itself depends on $\qoi$, and both densities on $(\dspace, \dborel)$ are evaluated at $\qlam$.
In the event that the map $\qoi$ is a bijection, then the consistency condition \eqref{eq:inverse-problem} defines a unique measure $\PP$ on $\pspace$ given the specification of an observed density.
However, there are many applications of interest where $\qoi$ fails to be a bijection, either due to differences in the dimensions of the parameter and data spaces, nonlinearities inherent in the model itself, or both.


The specific nuances of this relationship are discussed in \ref{chapter:02} and \ref{chapter:03} in greater detail.
Here, we simply note that the construction of \eqref{eq:update} requires only the forward-problem construction of $\predicted$, since $\initial$ and $\observed$ are given \emph{a priori}.
Additional properties are given in \ref{sec:properties} alongside the conditions for the existence and uniqueness of an update of the form given by \eqref{eq:inverse-problem}.



%%%%%%%%%%%%%%%%%%%



A typical Bayesian approach to an inverse problem focuses on first modeling epistemic uncertainties in data on a QoI obtained from a true, but unknown, parameter value, which we denote by $\paramref$.
This is in contrast to the SIP and its observation-consistent solutions that are defined as pullback measures of an observed probability measure on the QoI.
To help build intuition about these differences, we summarize key details about these inverse problems and their solutions before presenting an example that highlights differences in solutions.

%As mentioned in the introduction, the observation-consistent framework developed in \cite{BJW18a, BJW18b, BWY20} is designed to quantify aleatoric sources of uncertainty while the typical Bayesian framework \cite{0266-5611-7-5-003,
%  Kennedy_O_JRSSSB_2001,Tarantola_book, MNR07, CDS10,starktenorio,
%  AlexanderianPetraStadlerEtAl14, Bui-ThanhGhattas14, Ernst2014,
%  0266-5611-30-11-110301, ROM:CMW_2016,Stuart10,
%  cockayneoatessullivangirolami} is designed to quantify epistemic sources of uncertainty.
%These conceptual differences have significant impacts on the solutions to inverse problems formulated within these distinctive frameworks.
%We provide more details below to further clarify these impacts for the reader.
%An example is then used to illustrate the differences, which is also helpful for building intuition.
%Moreover, the details provided below play a vital role in Section~\ref{sec:estimation} where features of the observation-consistent framework are used to motivate its extension to point estimation problems.


%The Bayesian inversion framework  is perhaps the most
%popular approach in the UQ community for incorporating uncertainties in inverse solutions.
%As discussed in the introduction, the typical Bayesian framework focuses on first modeling epistemic uncertainties in data on a QoI obtained from a true, but unknown, parameter value, which we denote by $\paramref$.


\subsubsection{The stochastic inverse problem (SIP)}

We first define the concept of push-forward measures as solutions to the SFP mentioned in the introduction.
This helps frame the SIP more clearly as the direct inversion of the SFP.
\begin{definition}[Stochastic Forward Problem (SFP)]\label{def:forward-problem}
  Given an initial (i.e., initially specified) probability measure $\initialP$ on $\pspace$, the SFP is the determination of the push-forward probability measure $\predictedP$ on $\dspace$, defined by
\begin{equation*}
\predictedP(E) := \initialP(Q^{-1}(E)),
\end{equation*}
for all events $E\subset\dspace$.
\end{definition}

We often refer to the push-forward of the initial measure as the {\em predicted} measure since it may be constructed before any observed data are known.
This also helps to distinguish it from the {\em observed} measure used in the formulation of the SIP.
\begin{definition}[Stochastic Inverse Problem (SIP)]\label{def:inverse-problem}
Given an observed probability measure $\observedP$ on $\dspace$, the SIP is to determine a pullback probability measure, denoted $\mathbb{P}_\pspace$, on $\pspace$, which is observation-consistent in the sense that
\begin{equation}\label{eq:data-consistent}
\mathbb{P}_\pspace(Q^{-1}(E)) = \observedP(E),
\end{equation}
for all events $E\subset\dspace$.
\end{definition}

Unless the map $Q$ is a bijection, we do not expect that there is a unique $\mathbb{P}_\pspace$ solving the SIP, but rather there is a class of pullback measures that solve the SIP.
In \cite{BET+14}, a disintegration theorem \cite{Chang_Pollard} along with an ansatz is used to establish the existence of solutions to the SIP that are unique up to the choice of ansatz.
An algorithm is provided in \cite{BET+14} for explicitly approximating pullback measures by applying a specified ansatz to approximations of contour events, i.e., approximations of $Q^{-1}(E_i)$ where $\set{E_i}_{i\in\mathcal{I}}$ is a partition of $\dspace$.

In \cite{BJW18a}, an alternative density-based approach is presented that is computationally simpler to implement, scales well with increasing parameter dimension, and is stable with respect to perturbations in the initial and observed probability measures.
We refer the interested reader to \cite{BJW18a} for the theoretical and algorithmic details.
Here, we simply summarize the density-based solution to the SIP as:
\begin{equation}\label{eq:updated-pdf}
	\updated(\param) := \initial(\param)\frac{\observed(Q(\param))}{\predicted(Q(\param))}.
\end{equation}
The densities (i.e., Radon-Nikodym derivatives) $\initial$ and $\observed$ are associated with the specification of $\initialP$ and $\observedP$, respectively.
The density $\predicted$ is associated with the predicted measure $\predictedP$.
In other words, constructing $\updated$ as a solution to the SIP requires a solution to the SFP.

In order to ensure that $\updated$ is in fact a density, a predictability assumption is required \cite{BJW18a}.
A practical form of the predictability assumption is that there exists a constant $C>0$ such that $\observed(q)\leq C\predicted(q)$ for (almost every) $q\in\dspace$.
Conceptually, we interpret the predictability assumption as stating that we are able to predict all the of the observed data.
This also helps to frame the special role of $\initial$ in the SIP as compared to the role of the prior density used in the Bayesian inverse problem that is discussed below.
Specifically, $\initial$ allows us to perform (1) robust predictions, and (2) define a particular observation-consistent solution.
