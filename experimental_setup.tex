We want an estimated probability measure $\hat{\PP}_\pspace$ on the parameter space to converge with respect to $d_H$ to some reference measure $\PP_\pspace$ (either some known prior distribution taken as truth or another approximation deemed to be sufficiently resolved for the given application or computational budget).\footnote{However, we could also choose to interrogate the push-forward measures given by propagating the $\hat{\PP}_\pspace$ and $\PP_\pspace$ forward to a data space by a QoI map and taking the Hellinger distance on the resulting output space. 
This would measure the ability of the maps to reconstruct the output probability measure.}

All of our experiments follow the same structure:
\begin{itemize}
\item[[0-a]] Select $\qoi\in\qspace$ and define $\PP_{\dspace_\qoi}$ as a uniform distribution centered on a reference QoI value $Q(\paramref)$ for $\paramref$ taken as the midpoint of $\pspace$. 
Note that $\PP_{\dspace_\qoi}$ is exactly discretized with $M=1$ sample, so that 
\[
P_{\pspace, 1} = P_\pspace.
\]
\item[[0-b]] Create a regular grid of samples in $\pspace=[0,1]^n$ using $N_{\text{ref},i}$ equispaced points in each dimension. 
Define $\bar{N} := \prod N_{\text{ref},i}$.
Since $n$ is small in the numerical examples shown here, we chose $N_{\text{ref},i} = 200 \; \forall \; i$ in each example.
\item[[0-c]] Use Algorithm~\ref{alg:inv_density} to construct a reference solution $\PP_{\pspace,\bar{N}}\approx \PP_\pspace$.
\item[[1]] Generate $\set{S_k^{(n)}}_{n=1}^{50}$ sets of uniform i.i.d.~random samples where $N_k = 25, 50, 100, 200, \hdots, 6400$, and $n$ represents the number of repeated trials of a sample size $N_k$.
%, constructing $\set{\set{\VVV_k^{(j)}}_{k=1}^{50}}_{j=1}^{N}$ so that when we compute Hellinger distances on the approximate measures defined on each $\set{\VVV_k^{(j)}}{j=1}{N}$, we can reduce the variance in our expected Hellinger distance values for each instance of $N$. Note that we experimented with using more trials and found the variance in expected Hellinger distances was sufficiently low with as few as twenty trials for the maps under consideration herein.
%\item[[3]] For every trial $T$ and $N$ value (including $\bar{N}$), the reference parameter $\lambda = (\lambda_1, \lambda_2) = (0.5, 0.5)$ is mapped by $Q$ to $\dspace_\qoi = Q(\pspace)$.
%\item[3] A uniform distribution with support $[Q(\lambda_1) - 0.05, Q(\lambda_1) + 0.05] \times [Q(\lambda_2) - 0.05, Q(\lambda_2) + 0.05]$ is defined on $\dspace_\qoi$, representing equal uncertainty in each component of our measured functional values.
\item[[2]] Solve the SIPs using Algorithm~\ref{alg:inv_density} to construct $\set{\PP_{\pspace,M,N}^{(n)}}_{n=1}^{50}$.
\item[[3]] Use $1E5$ i.i.d.~random samples in the Monte Carlo step of Algorithm~\ref{alg:hellinger_disc} to estimate $\set{d_H^2( \PP_{\pspace,M,N}^{(n)}, \PP_{\pspace,\bar{N}})}_{n=1}^{50}$.
\item[[4]] Average over all trials $n$ for each $N$ to estimate the {\em expected} Hellinger distance for $N$ samples and analyze convergence to $\PP_{\pspace,\bar{N}}$. 
\item[[5]] Repeat steps [0-a]--[4] for each $\qoi\in\qspace$ under consideration. 
\end{itemize}
To isolate the effect of skewness on our ability to approximate sets with finite sampling, we choose our maps so that they preserve the sizes of sets between $\pspace$ and $\dspace$ under the push-forward measure given in Eq.~\eqref{eq:dataspace_pushforward_measure}. 

