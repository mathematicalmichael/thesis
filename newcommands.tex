%%%%%%%%%%%%%%%%%%%%%%%%%%%%%%%%%%%%%%%%%%%%%%%%%%%%%%%%%%%%%%%%%%%%%%%%%%%%%%%%
% newcommands.tex

% itemize with the explaining text left justified afer the item
% Example:
%   \itembox{CNR} The Control net reduction
%   \itembox{A} Another item
% renders as
% * CNR   The Control net reduction
% * A     Another item
\newcommand{\itembox}[1]{\item {\makebox[0.25in]{#1 \hfill}}}

% Abbreviations 
\newcommand{\ie}{{\itshape i.e.}}
\newcommand{\eg}{{\itshape e.g.}}

% Bold symbols
\newcommand{\bs}[1]{\boldsymbol{#1}}

% Cardinality
\newcommand{\card}[1]{n\left(#1\right)} %{\left|#1\right|}
\newcommand{\set}[1]{\left\{#1\right\}}

% Math Operators
\DeclareMathOperator*{\argmin}{arg\,min}
\DeclareMathOperator*{\argmax}{arg\,max}
%\newcommand{\norm}[1]{\left\Vert#1\right\Vert}

% Standard Probability
\DeclareMathOperator{\E}{\mathbb{E}}
\newcommand{\PP}{\mathbb{P}}
\newcommand{\RR}{\mathbb{R}}
\newcommand{\RP}{\mathbb{R}^P}
\newcommand{\CC}{\mathcal{C}}
\newcommand{\OO}{\mathcal{O}}
\renewcommand{\LL}{\mathcal{L}}
\newcommand{\VV}{\mathcal{V}}
\newcommand{\BB}{\mathcal{B}}
\newcommand{\RS}{\mathbb{R}^S}
\newcommand{\RD}{\mathbb{R}^D}
\newcommand{\sa}{\sigma\text{-algebra}}
\newcommand{\Chi}{\mbox{\LARGE$\chi$}}

\newcommand{\Rho}{\mbox{\LARGE$\rho$}}
\newcommand{\Nu}{\mbox{\Large$\nu$}}

% Data-Consistent Inversion Framework

\newcommand{\param}{\lambda}
\newcommand{\pspace}{{\Lambda}}
\newcommand{\pmeas}{\mu_{\pspace}}
\newcommand{\pborel}{\BB_{\pspace}}
\newcommand{\Pspace}{(\pspace, \pborel, \pmeas)}

\DeclareMathOperator*{\initialP}{\PP_{in}}
\DeclareMathOperator*{\initial}{\pi_{in}}
\DeclareMathOperator*{\updatedP}{\PP_{up}}
\DeclareMathOperator*{\updated}{\pi_{up}}

\DeclareMathOperator*{\obs}{\boldsymbol{o}}
\DeclareMathOperator*{\data}{\boldsymbol{d}}
\newcommand{\dspace}{\mathcal{D}}
\newcommand{\dmeas}{\mu_{\dspace}}
\newcommand{\dborel}{\BB_{\dspace}}
\newcommand{\Dspace}{(\dspace, \dborel, \dmeas)}


\newcommand{\qspace}{\mathcal{Q}}
\newcommand{\lam}{\left ( \param \right ) }
\newcommand{\qoi}{Q}
\newcommand{\qlam}{\qoi\lam}
\DeclareMathOperator*{\M}{\mathcal{M}}

\DeclareMathOperator*{\observedP}{\PP_{ob}}
\DeclareMathOperator*{\observed}{\pi_{ob}}
\DeclareMathOperator*{\predictedP}{\PP_{pr}}
\DeclareMathOperator*{\predicted}{\pi_{pr}}

\DeclareMathOperator*{\dciP}{\updatedP = \initialP \frac{\observedP}{\predictedP}}
\DeclareMathOperator*{\dciD}{\updated = \initial \frac{\observed}{\predicted}}
\DeclareMathOperator*{\dci}{\updated\lam = \initial\lam \frac{\observed\qlam}{\predicted\qlam}}

%%%%%%%%%%%%%%%%%%%%%%%%%%%%%%%%%%%%%%%%%%%%%%%%%%%%%%%%%%%%%%%%%%%%%%%%%%%%%%%%

