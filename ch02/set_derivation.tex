To properly summarize the Stochastic Inverse Problem (SIP) and desired solution, we define several measure/probability spaces and refer to the schematic given in Figure \ref{fig:scheme}\---borrowed from \cite{BM17}\---in order to illustrate the steps and spaces required in the formulation and solution of the SIPs we consider herein.
For a more extensive review, we refer the reader to \cite{BBE11}, \cite{BES12}, \cite{BET+14}, and \cite{BM17}.

%%%%%%%%%%%%%%%%%%%%%%%%
\begin{figure}[!h]
\begin{equation}
\underbrace{
\underbrace{
\overbrace{
 \Pspace \xmapsto{\  Q \ } \Dspace
  \xmapsto{\ \dataP \ } (\dspace, \dborel, \dataP)
 }^{
 \text{(S1): Stochastic Inverse Problem (SIP)}
 }
 \xmapsto{\ Q^{-1} \ } (\pspace, \cborel, \contourP)
 }_{
 \text{(S2): Solution to SIP Satisfying Eq. \eqref{eq:dataspace_pushforward_measure}}}
 \xmapsto{\ \set{\PP_\ell}_{\ell\in\mathcal{L}} \ } (\pspace, \pborel, \paramP)
 }
 _{
 \text{(S3): Unique Solution to SIP by Eq.~\eqref{eq:disintegration_measure} and Ansatz}
 }
\end{equation}
\caption{The first step (S1) defines (i)~the formulation of the SIP by specification of the model, (ii)~the measure spaces of parameters and (iii)~observable outputs, and (iv)~the probability measure on the latter. The second step (S2) defines a unique solution to the SIP on the space $\pspace$ equipped with the contour $\sa$ $\cborel$ using the definition of the push-forward measure. In (S3), the Disintegration Theorem and and Ansatz are applied to define a unique solution on the space of interest $(\pspace, \pborel)$ equipped with a probability measure $\paramP$.}
\label{fig:scheme}
\end{figure}


The initial measure/probability spaces involved in the formulation of the SIP are summarized in step (S1) of Fig.~\ref{fig:scheme}, starting with measure space $\Pspace$.

The assumption that $\qoi$ is at least piecewise-differentiable implies the measurability of the QoI map, so that the space $\dspace$ induced by $\qoi$ is equipped with the Borel $\sa$ $\dborel$ [TK - cite textbook].
The ``push-forward'' measure $\dmeas$ on ${(\dspace, \dborel)}$ is defined as
\begin{equation}\label{eq:dataspace_pushforward_measure}
\dmeas (A) = \int_A \, d\dmeas := \int_{\qoi^{-1}(A)} \, d\pmeas = \pmeas \left (\qoi^{-1}(A) \right ) \quad \forall \;  A\in\dborel,
\end{equation}
which defines the measure space $\Dspace$\footnote{When referring to properties of the data space that are not unique to the choice of map used to induce $\dspace$, we will drop the subscript notation and assume the dependence is understood, as expressed in Fig.~\ref{fig:scheme}.}.
The push-forward distribution $\partial \dataP / \partial \mu_D$ is referred to as the \emph{predicted} distribution and is given by the Radon-Nikodym derivative with respect to (the Lebesgue) volume measure $\mu_D$ on ${(\dspace, \dborel)}$

The final step in (S1) involves the specification of a probability measure $\dataP$ on ${(\dspace, \dborel)}$ to model the uncertainty in data.
We colloquially refer to the distribution (given by the Radon-Nikodym derivative $\partial \dataP / \partial \pmeas$) as the \emph{observed} distribution, where $\pmeas$ is taken to be the (Lebesgue) volume measure on ${(\pspace, \pborel)}$.
This leads to the following SIP: determine a probability measure $\paramP$ on ${(\pspace, \pborel)}$ such that the push-forward measure of $\paramP$ matches $\dataP$.

In other words, determine a $\paramP$ satisfying
\begin{equation}\label{eq:inverse_measure}
\paramP \left ( \qoi^{-1}(E)\right ) = \PP_{\dspace}(E) \; \forall \, E \in \dborel.
\end{equation}

We call any such solution $\paramP$ to Eq.~\eqref{eq:inverse_measure} a (measure-theoretic) solution to the SIP.
This equation implies that any solution is uniquely determined on the induced contour $\sa$
\begin{equation}\label{eq:contour_sa}
\cborel = \set{\qoi^{-1}(E) : E \in \dborel } \subset \pborel,
\end{equation}
which is summarized as step (S2) of Fig.~\ref{fig:scheme}.

However, for sets $A \in \pborel \setminus \cborel$, more information is required than is provided in Eq.~\eqref{eq:inverse_measure} in order to determine $\paramP (A)$.
By the Implicit Function Theorem, if $\data \in C^1 (\pspace)$ and we let $\data\in\dspace$ be a fixed datum, $\qoi^{-1}(q)$ exists as a $(\nparams-\ndata)$\--dimensional manifold (possibly piecewise-defined) that we refer to as a \emph{generalized contour} \cite{BET+14}.
These generalized contours can be indexed by a manifold (also possibly piecewise-defined) of dimension $\ndata$ called a \emph{transverse parameterization} that intersects each contour once and only once.
In \cite{BET+14}, it is shown that transverse parameterizations are guaranteed to exist but are in general not unique.

We let $\LL$ denote any particular transverse parameterization.
Each $\ell\in\LL$ corresponds to a unique generalized contour $\CC_\ell \in \pspace$ and each point $\param\in\pspace$ belongs to a unique $\CC_\ell\in\pspace$.
Thus, a transverse parameterization defines a bijection between the manifold $\LL$ and the partitioning of $\pspace$ into generalized contours.
The induced $\sa$ $\cborel$ and this bijection can then be used to define the measurable space $(\LL, \BB_\LL)$.

We denote the projection map $P_\LL : \pspace \to \LL$, and let $\set{\CC_\ell}_{\ell\in\LL}$ represent the family of generalized contours indexed by $\LL$, yielding the associated family of measurable spaces $\set{\left ( \CC_\ell, \BB_{\CC_\ell} \right )}_{\ell\in\LL}{}$.
A Disintegration Theorem [TK - cite] is then leveraged to define a unique decomposition for any $\paramP$ defined on $(\pspace, \pborel)$ as a (marginal) probability measure $\PP_\LL$ on $(\LL, \BB_\LL)$ and a family of (conditional) probability measures $\set{\PP_\ell}_{\ell\in\LL}$ on $\set{\left ( \CC_\ell, \BB_{\CC_\ell} \right )}_{\ell\in\LL}$ such that
\begin{equation}\label{eq:disintegration_measure}
\paramP (A) = \int_{P_\LL(A)} \left ( \int_{P_{\LL}^{-1} (\ell) \cap A}\, d\PP_\ell(\param) \right )\, d\PP_\LL (\ell), \; \forall \; A \in \pborel
\end{equation}

The uniqueness of a probability measure $\paramP$ on ${(\pspace, \cborel)}$ satisfying Eq.~\eqref{eq:inverse_measure} implies the uniqueness of the marginal probability measures $\PP_\LL$ for any particular specification of $\dataP$ on ${(\dspace, \dborel)}$.
The disintegration of Eq.~\eqref{eq:disintegration_measure} implies that a specification of a family of conditional probability measures $\set{P_\ell}_{\ell\in\LL}$ gives us a unique solution to the SIP on ${(\CC_\ell, \BB_{\CC_\ell})}$.

However, the conditional measures cannot be determined by solely by the specification of $\data\in\dspace$.
We follow the work of \cite{BET+14} and adopt the \emph{standard ansatz} determined by the disintegration of the volume measure $\pmeas$ to compute probabilities of sets contained within contour events.
The standard ansatz is given by
\begin{equation}\label{eq:standard_ansatz}
\PP_\ell = \mu_{\CC_\ell} / \mu_{\CC_\ell}(\CC_\ell), \; \forall \; \ell \in \LL,
\end{equation}
where $\mu_{\CC_\ell}$ is the disintegrated volume measure on generalized contour $\CC_\ell$.
Thus, we have defined a unique solution to the SIP on ${(\pspace, \pborel)}$, completing step (S3) in Fig.~\ref{fig:scheme}.
