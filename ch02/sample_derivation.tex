In [TK - reference] it was shown that an equivalent derivation to the same set-based solution to the SIP presented in \ref{sec:ch02-set} could be achieved with the following form:

\begin{equation}
\dciP
\end{equation}


This equation presents on the left-hand side the solution to the SIP, referred to as the updated measure $\updatedP$, which is equal to a scaling of an initial probabaility measure $\initialP$ by a ratio of observed $\observedP$ to predicted $\predictedP$ measures.
Taking the Radon-Nikodym derivatives of each of the respective terms, we can arrive at a more natural distribution-based description of the solution:

\begin{equation}
\begin{split}
\dci\\
\dciD
\end{split}
\end{equation}

The initial density $\initial$ encodes the information that the ansatz played in \ref{sec:ch02-set}, which is to assign relative probabilities among points which belong to the same equivalence class of solutions.
