\chapter{\uppercase{Background on Data-Consistent Inversion} \label{chapter:02}}

\section{Notation, Terminology, and Assumptions}

\
\section{Set-Based Inversion for Measures}

\
\section{Sample-Based Inversion for Measures \label{sec:ch02:sample}}

\
\section{Software Contributions}

\subsection{Background and Motivation}
The open-source software package BET was developed actively from 2012-2015 as part of research performed under grant [TK grant-DOE].
It was originally written in Python 2.7 and is administered by the Computational Hydrology Group at the University of Texas: Austin through their UT-CHG GitHub group [TK - cite Github]. 
The purpose of this open-source software package was to implement the methods first described in [TK - cite BET papers] for the description and solution of stochastic inverse problems. 

In the intermittent years since its original publication in [TK - date of first release, cite Github], the BET package has seen two major releases and the incorporation of several submodules (e.g. the functions in {\tt sensitivity} implement much of the original research performed by Dr. Walsh [TK - cite Scott]). 
Since the last major release [TK - cite latest release], the Python community announced the end of long-term support for Python 2 [TK - cite announcement]. 
Several of the dependendencies in BET have been actively developed in Python 3 with no updates to the Python 2 analogs, which suggested that BET should likely undergo the same transition.

The work summarized in Section~\ref{sec:ch02:sample} was implemented in Python 3 independently by the author through the release of the ConsistentBayes package.
Since that code was used for many of the results that constituted the preliminary results for this work, it made very litle sense to re-implement them in Python 2 for BET given the recent trends in community development. 
With funding made available through NSF [TK - cite grant], the opportunity to upgrade BET to Python 3 was the most sensible choice. 

The upgrade to Python 3.6 began in January 2019 as a first step to incorporate the new sample-based method into BET.
Major release [TK - put in release] was designed to provide backwards-compatibility with the Python 2.7 version. 
Future installation specifications would not limit the versions of some core dependencies in order to provide backwards-compatibility with Python 2 (e.g. {\tt numpy}, {\tt scipy}) since this would likely downgrade previously installed software for end-users. 

Since unit-testing with {\tt nose} was always a priority for the developers of BET, tests for plotting were removed from the release since these were most likely to break with upgrades to {\tt matplotlib}.
Basic plotting functionality is demonstrated in iPython notebooks, which have seen an exponential growth rate on GitHub, which can be edited by the end-user to work with different plotting library versions and backends.
This is in contrast to the previous releases, which packages plotting as a sub-module that was subject to passing tests.   




\
\section{Illustrative Examples}


