%%%%%%%%%%%%%%%%%%%%%%%%%%%%%%%%%%%%%%%%%%%%%%%%%%%%%%%%%%%%%%%%%%%%%%%%%%%%%%%%
% newcommands.tex

% itemize with the explaining text left justified afer the item
% Example:
%   \itembox{CNR} The Control net reduction
%   \itembox{A} Another item
% renders as
% * CNR   The Control net reduction
% * A     Another item
\newcommand{\itembox}[1]{\item {\makebox[0.25in]{#1 \hfill}}}

% Abbreviations
\newcommand{\ie}{{\itshape i.e.}}
\newcommand{\eg}{{\itshape e.g.}}

% Bold symbols
\newcommand{\bs}[1]{\boldsymbol{#1}}

% Cardinality
\newcommand{\card}[1]{n\left(#1\right)} %{\left|#1\right|}
\newcommand{\set}[1]{\left\{#1\right\}}
\newcommand{\nullset}{\varnothing}

% Math Operators
\DeclareMathOperator*{\argmin}{arg\,min}
\DeclareMathOperator*{\argmax}{arg\,max}
\newcommand{\abs}[1]{\left\vert#1\right\vert}
\newcommand{\norm}[1]{\left\Vert#1\right\Vert}
\newcommand{\mat}[2]{\left[\begin{array}{#1}#2\\ \end{array}\right]}

% Standard Probability
\DeclareMathOperator{\E}{\mathbb{E}}
\newcommand{\PP}{\mathbb{P}}
\newcommand{\PPspace}{\mathcal{P}}


% indices into sets/spaces
\newcommand{\iparam}{i} % index into parameter space (num samples)
\newcommand{\imesh}{h} % index into mesh/model refinement sequence
\newcommand{\iobs}{k} % index into QoI
\newcommand{\idisc}{k} % index into output discretization
\newcommand{\idata}{j} % index into data (target)

% Sets/Spaces
\newcommand{\RR}{\mathbb{R}}
\newcommand{\OO}{\mathcal{O}} % sets of indices for set-based inversion alg
\renewcommand{\LL}{\mathcal{L}} % transverse parameterization
\newcommand{\VV}{\mathcal{V}} % voronoi
\newcommand{\BB}{\mathcal{B}} % borel

\newcommand{\CC}{\mathcal{C}} % contour symbol
\newcommand{\cborel}{\CC_{\pspace}} % contour sigma algebra
\DeclareMathOperator*{\contourP}{\PP_{\cborel}}

\DeclareMathOperator*{\paramP}{\PP_{\pspace}}
\DeclareMathOperator*{\dataP}{\PP_{\dspace}}

\newcommand{\nparams}{P}
\newcommand{\RP}{\RR^\nparams} % parameter space dimensions
\newcommand{\nsamps}{N} % num samples

\newcommand{\ndata}{D} % num data points collected for each observable
\newcommand{\RD}{\RR^\ndata} % data space dimensions

% Sample-Based
\newcommand{\nobs}{O} % number of observables
\newcommand{\RO}{\RR^\nobs} % qoi data space dimensions

% Set-Based
\newcommand{\ndiscs}{M} % number of cells used to discretize data space

\newcommand{\sa}{\sigma\text{-algebra}}
\newcommand{\Chi}{\mbox{\LARGE$\chi$}}

\newcommand{\Rho}{\mbox{\LARGE$\rho$}}
\newcommand{\Nu}{\mbox{\Large$\nu$}}


% Data-Consistent Inversion Framework
\newcommand{\dimP}{P}
\newcommand{\dimD}{D}
\newcommand{\param}{\lambda}
\newcommand{\paramref}{\param^\dagger}
\newcommand{\pspace}{{\Lambda}}
\newcommand{\pmeas}{\mu_{\pspace}}
\newcommand{\pborel}{\BB_{\pspace}}
\newcommand{\Pspace}{(\pspace, \pborel, \pmeas)}

\DeclareMathOperator*{\initialP}{\PP_{in}}
\DeclareMathOperator*{\initial}{\pi_{in}}
\DeclareMathOperator*{\updatedP}{\PP_{up}}
\DeclareMathOperator*{\updated}{\pi_{up}}

\DeclareMathOperator*{\obs}{\boldsymbol{o}}
\DeclareMathOperator*{\data}{\boldsymbol{d}}
\newcommand{\dspace}{\mathcal{D}}
\newcommand{\dmeas}{\mu_{\dspace}}
\newcommand{\dborel}{\BB_{\dspace}}
\newcommand{\Dspace}{(\dspace, \dborel, \dmeas)}


% Labeled QoI maps
\newcommand{\qoiA}{\qoi^{(a)}}
\newcommand{\qoiB}{\qoi^{(b)}}
\newcommand{\qoiC}{\qoi^{(c)}}
\newcommand{\dspaceA}{\dspace_{\qoiA}}
\newcommand{\dspaceB}{\dspace_{\qoiB}}

\newcommand{\qspace}{\mathcal{Q}}
\newcommand{\lam}{\left ( \param \right ) }
\newcommand{\qoi}{Q}
\newcommand{\qlam}{\qoi\lam}
\newcommand{\M}{\mathcal{M}}

\DeclareMathOperator*{\observedP}{\PP_{ob}}
\DeclareMathOperator*{\observed}{\pi_{ob}}
\DeclareMathOperator*{\predictedP}{\PP_{pr}}
\DeclareMathOperator*{\predicted}{\pi_{pr}}

\DeclareMathOperator*{\dciP}{\updatedP = \initialP \frac{\observedP}{\predictedP}}
\DeclareMathOperator*{\dciD}{\updated = \initial \frac{\observed}{\predicted}}
\DeclareMathOperator*{\dci}{\updated\lam = \initial\lam \frac{\observed\qlam}{\predicted\qlam}}

%%%%%%%%%%%%%%%%%%%%%%%%%%%%%%%%%%%%%%%%%%%%%%%%%%%%%%%%%%%%%%%%%%%%%%%%%%%%%%%%
% Theorems
\theoremstyle{plain}
\newtheorem{thm}{Theorem}
\newtheorem{cor}{Corollary}
\newtheorem{lem}{Lemma}
\newtheorem{prop}{Proposition}
\theoremstyle{definition}
\newtheorem{defn}{Definition}
%\theoremstyle{remark}
\newtheorem{rem}{Remark}
%\theoremstyle{definition}
\newtheorem{ex}{Example}
\numberwithin{equation}{section}
\newtheorem{prob}{Problem}
\numberwithin{equation}{section}
%
