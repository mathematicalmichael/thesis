\subsection{Background and Motivation}
The open-source software package BET was developed actively from 2012-2015 as part of research performed under grant [TK grant-NSF+DOE].
It was originally written in Python 2.7 and is administered by the Computational Hydrology Group at the University of Texas: Austin through their UT-CHG GitHub group [TK - cite Github].
The initial purpose of this open-source software package was to implement the methods first described in [TK - cite BET papers] for the description and solution of stochastic inverse problems summarized in Section~\ref{sec:ch02-set}.

In the intermittent years since its original publication in [TK - date of first release, cite Github], the BET package has seen two major releases and the incorporation of several sub-modules (e.g. the functions in {\tt sensitivity} implement much of the original research performed by Dr. Walsh [TK - cite Scott]).


\subsection{Upgrades, Updates, and Features}
Since the last major release [TK - cite latest release], the Python community announced the end of long-term support for Python 2 [TK - cite announcement].
Several of the dependencies in BET have been actively developed in Python 3 with no updates to the Python 2 analogs, which suggested that BET should likely undergo the same transition.

The work summarized in Section~\ref{sec:ch02-sample} was implemented in Python 3 independently by the author through the release of the ConsistentBayes package.
Since that code was used for many of the preliminary results for this work, it made very little sense to re-implement them in Python 2 for BET given the recent trends in community development.
With funding made available through the NSF [TK - cite grant], the opportunity to upgrade BET to Python 3 was the most sensible choice.


\subsubsection{Version 2.1.0}
The upgrade to Python 3.4+ began in January 2019 as a first step to incorporate the new sample-based method into BET.
It was completed in late February.
Major (minor? version? TK) release 2.1.0 [TK - put in release] was designed to provide backwards-compatibility with the Python 2.7 version.
Future installations (starting in 2020) will not limit the versions of some core dependencies in order to accommodate backwards-compatibility with Python 2 (e.g. {\tt numpy}, {\tt scipy}), since this would likely downgrade previously installed software for end-users.


\subsubsection{Version 2.2.0}
Several releases of BET (after the upgrade to Python 3 in v2.1.0), incorporated developments that will be discussed herein.
For


\subsubsection{Version 2.2.1}
Major bugfix for parallel testing allowed tests to pass for more than 2 processors.
For some tests, this involved changing the setup parameters to ensure the problem was large enough to break up onto up to 8 processors.
For others, siginficant changes had to be made to structure to allow for proper saving and loading of files in parallel.


\subsubsection{Version 2.3.0}
This release incorporated the sampling-based approach discussed in Section~\ref{sec:ch02-sample}.


\subsection{Examples in BET}
Basic plotting functionality of BET is demonstrated in iPython notebooks [TK - some kind of citation here], which have seen an exponential growth rate on GitHub, and can be edited by the end-user to work with different plotting library versions and backends.
These notebooks were originally created to reflect the example suite in BET (which were {\tt .py} files), but later they were migrated into a separate repository BET-examples to allow for better organization.
In the new framework, each notebook functions as an independent example.
Several of these notebooks were adapted from the example code in this thesis repository.

