\subsection{Problem Formulation and Solution}
%%%%%%%%%%%%%%%%%%%%%%%%%%%%%%%%%%%%%%%%%%%%%%%%%%%%%%%%%%%
\begin{frame}[t]

\begin{defn}[Stochastic Forward Problem (SFP)]\label{defn:forward-problem}
  Given a probability measure $\PP_\pspace$ on $(\pspace, \pborel)$, and QoI map $\qoi$, the \emph{stochastic forward problem} is to determine a measure, $\PP_\dspace$, on $(\dspace, \dborel)$ that satisfies
  \begin{equation}\label{eq:forward-problem}
    \PP_\dspace (E) = \PP_\pspace \left ( \qoi^{-1}(E) \right ), \; \forall \; E \in \dborel.
  \end{equation}
\end{defn}

\end{frame}

%%%%%%%%%%%%%%%%%%%%%%%%%%%%%%%%%%%%%%%%%%%%%%%%%%%%%%%%%%%
\begin{frame}[t]

\begin{defn}[Stochastic Inverse Problem (SIP)]\label{defn:inverse-problem}
  Given a probability measure, $\PP_\dspace$, on $(\dspace, \dborel)$ the \emph{stochastic inverse problem} is to determine a probability measure, $\PP_\pspace$, on $(\pspace, \pborel)$ satisfying
  \begin{equation}\label{eq:inverse-problem}
    \PP_\pspace (\qoi^{-1}(E)) = \PP_\dspace(E), \; \forall \; E \in \mathcal{B}_\dspace.
  \end{equation}
\end{defn}

The above is known as the \emph{consistency condition}.
\end{frame}

%%%%%%%%%%%%%%%%%%%%%%%%%%%%%%%%%%%%%%%%%%%%%%%%%%%%%%%%%%%
\begin{frame}[t]

\begin{defn}[Consistent Solution]\label{defn:consistent-solution}
  \noindent Any probability measure $\PP_\pspace$ satisfying \eqref{eq:inverse-problem} is referred to as a \emph{consistent solution} to the inverse problem, and \eqref{eq:inverse-problem}.

  If $\PP_\pspace$ or $\PP_\dspace$ absolutely continuous w.r.t $\pmeas$ or $\dmeas$, resp, then we write

  \begin{equation*}
    \pp_\pspace := \frac{d\PP_\pspace}{d\pmeas} \;\text{ or }\; \pp_\dspace := \frac{d\PP_\dspace}{d\dmeas}
  \end{equation*}
  to denote the Radon-Nikodym derivatives (i.e., pdfs) of $\PP_\pspace$ and $\PP_\dspace$, resp.

  In such a case, we can rewrite \eqref{eq:forward-problem} and \eqref{eq:inverse-problem} using these pdfs:
  \begin{equation*}
  \PP_\pspace (\qoi^{-1}(E)) = \int_{\qoi^{-1}(E)} \pp_\pspace \lam \, d\pmeas = \int_E \pp_\dspace \Q \, d\dmeas = \PP_\dspace(E), \; \forall \; E \in \mathcal{B}_\dspace.
  \end{equation*}

\end{defn}

\end{frame}

%%%%%%%%%%%%%%%%%%%%%%%%%%%%%%%%%%%%%%%%%%%%%%%%%%%%%%%%%%
\begin{frame}[t]{Perspectives}
\begin{itemize}
	\item We seek the $\updatedP$ whose push-forward measure matches $\observedP$
	\item In measure-theoretic terms, $\updatedP$ is a pull-back measure of $\observedP$

	\begin{defn}[Observed Density]\label{defn:obsden}
		The density $\observed$ in \eqref{eq:inv} represents the uncertainty in QoI data.
	\end{defn}

	\begin{defn}[Initial Density]\label{defn:initialden}
		$\initial$ encodes prior beliefs about $\param$'s (before evidence is accounted for)
	\end{defn}

\end{itemize}

\end{frame}

%%%%%%%%%%%%%%%%%%%%%%%%%%%%%%%%%%%%%%%%%%%%%%%%%%%%%%%%%%
\begin{frame}[t]{Summarizing}
\begin{itemize}
	\item ``Push-forward'' \textbf{initial} beliefs using $\qoi$ to \textbf{compare} to \textbf{observed} (data)
	\item Solve forward problem to construct solution to inverse problem
	\item The push-forward density of $\initial$ under the map $\qoi$ is denoted by $\predicted$

	\begin{defn}[Predicted Density]\label{defn:predicted}
		$\predicted$ is given as the Radon-Nikodym derivative (with respect to $\mu_\dspace$) of the push-forward probability measure defined by:
		\begin{equation}\label{eq:pred}
			\predictedP (E)  = \initialP \left ( \qoi^{-1}(E) \right ), \; \forall \; E \in \dborel.
		\end{equation}
	\end{defn}

\end{itemize}

\end{frame}

%%%%%%%%%%%%%%%%%%%%%%%%%%%%%%%%%%%%%%%%%%%%%%%%%%%%%%%%%%
\begin{frame}[t]{The Updated Density solves the SIP}
These definitions are combined to form the \textbf{updated density}:
\begin{equation}\label{eq:up}
\updated \lam = \initial \lam \frac{\observed \qlam }{\predicted \qlam }, \; \param, \in \pspace.
\end{equation}

\begin{itemize}
	\item $\observed$ and $\predicted$ defined on $(\dspace, \dborel)$ are evaluated at $\qlam$
	\item The map $\qoi$ impacts the structure of the update
	\item $\dspace$ itself depends on $\qoi$
	\item Primary effort in solving for $\updated$ (in \eqref{eq:up}) requires constructing $\predicted$
	\item This is because $\initial$ and $\observed$ are given \emph{a priori} (often parametric)
	\item Updated derived through use of Disintegration Theorem in \cite{BJW18}
	\item Existence and Uniqueness given a predictability assumption
\end{itemize}
\end{frame}



\subsection{Properties and Assumptions of the Update}
%%%%%%%%%%%%%%%%%%%%%%%%%%%%%%%%%%%%%%%%%%%%%%%%%%%%%%%%%%
\begin{frame}[t]
\begin{assumption}[Predictability Assumption]\label{as:pred}
	The measure associated with $\observed$ is absolutely continuous with respect to the measure associated with $\predicted$.
\end{assumption}


The requirement is guaranteed if the following is satisfied:

\begin{equation}\label{eq:pred}
	\exists \; C>0 \text{ s.t. } \observed (d) \leq C \predicted (d) \text{ for a.e. } d\in \dspace,
\end{equation}

where $d = \qlam$ for some $\param \in \pspace$.
By \cite{BJW18}, if \eqref{as:pred} holds, we have: 

\begin{theorem}[Existence and Uniqueness]
	For any set $A\in \pborel$, the solution $\updatedP$ given defined by
	\begin{equation}\label{eq:cb_sol}
		\updatedP (A) = \int_\dspace \left (  \int_{\pspace \in \qoi^{-1}(d)}  \initial\param \frac{\observed(d)}{\predicted(d)} \, d\mu_{\pspace, d} \param \right ) \, d\mu_\dspace(d), \; \forall \; A \in \pborel
	\end{equation}

	is a consistent solution, and is unique up to choice of $\initialP$ on $(\pspace, \pborel)$.
\end{theorem}

\end{frame}


%%%%%%%%%%%%%%%%%%%%%%%%%%%%%%%%%%%%%%%%%%%%%%%%%%%%%%%%%%
\begin{frame}[t]{Stability}

All the stability and convergence results presented are with respect to:
\vspace{0.5in}

\begin{defn}{Total Variation / Statistical Distance}
	\begin{equation}\label{eq:tv}
		d_{\text{TV}} (\PP_f, \PP_g) := \int \abs{f - g} \, d\mu,
	\end{equation}
where $f,g$ are the densities (Radon-Nikodym derivatives with respect to $\mu$) associated with measures $\PP_f, \PP_g$, respectively.
\end{defn}

\end{frame}

%%%%%%%%%%%%%%%%%%%%%%%%%%%%%%%%%%%%%%%%%%%%%%%%%%%%%%%%%%
\begin{frame}[t]

\begin{defn}[Stability of Updates I]\label{defn:stableobs}
	We say that $\updatedP$ is \emph{stable} with respect to perturbations in $\observedP$ if for all $\eps > 0$, there exists a $\delta > 0$ such that
	\begin{equation}
		d_{\text{TV}} (\observedP, \widehat{\observedP}) < \delta \implies d_{\text{TV}} (\updatedP, \widehat{\updatedP}) < \eps.
	\end{equation}
\end{defn}

\vspace{0.5in}
In \cite{BJW18}, it is shown that $d_{\text{TV}} (\widehat{\updatedP}, \updatedP) = d_{\text{TV}} (\widehat{\observedP}, \observedP)$, implying that:
\vspace{0.5in}

\begin{theorem}
	$\updatedP$ is stable with respect to perturbations in $\observedP$.
\end{theorem}

\end{frame}

%%%%%%%%%%%%%%%%%%%%%%%%%%%%%%%%%%%%%%%%%%%%%%%%%%%%%%%%%%
\begin{frame}[t]
\begin{defn}[Stability of Updates II]\label{defn:stableinitial}
Let $\sett{\PP_{\pspace, d}}{d\in\dspace}{}$ and $\sett{\widehat{\PP_{\pspace, d}}}{d\in\dspace}{}$ be the conditional probabilities defined by the disintegration of $\initialP$ and $\widehat{\initialP}$, respectively.

We say that $\updatedP$ is \emph{stable} with respect to perturbations in $\initialP$ if for all $\eps > 0$, there exists a $\delta > 0$ such that for almost every $d\in\supp(\observed)$,
\begin{equation}\label{eq:stableinitial}
d_{\text{TV}} (\PP_{\pspace, d}, \widehat{\PP_{\pspace, d}}) < \delta \implies d_{\text{TV}} (\updatedP, \widehat{\updatedP}) < \eps.
\end{equation}
\end{defn}

\begin{theorem}
$\updatedP$ is stable with respect to perturbations in the initial.
\label{thm:stableinitial}
\end{theorem}


\end{frame}

%%%%%%%%%%%%%%%%%%%%%%%%%%%%%%%%%%%%%%%%%%%%%%%%%%%%%%%%%%
\begin{frame}[t]{Properties of the Updated Density}
\begin{itemize}

	\item Taken together, these stability results provide assurances that the updated we obtain is accurate up to the level of experimental error polluting $\observed$ and error in incorrectly specifying initial assumptions.
	\item Given that specifying the definition of a ``true'' initial is somewhat nebulous, we are less interested in the consequences of the latter conclusion.
	\item Generating samples from $\predicted$ requires a numerical approximation to $\predicted$, which introduces \textbf{additional errors} in $\predicted$.

\end{itemize}

\end{frame}


\subsection{Numerical Approximation and Sampling}
%%%%%%%%%%%%%%%%%%%%%%%%%%%%%%%%%%%%%%%%%%%%%%%%%%%%%%%%%%
\begin{frame}[t]
\begin{itemize}
	\item Let $\widehat{\predicted}$ be a computational approximation to $\predicted$ and $\widehat{\updated}$ the associated approximate updated $\updated$
	\item The conditional densities from the Disintegration theorem are 
\[
\frac{\widehat{d\PP_{\pspace, d}}}{d\mu_{\pspace, d}\lam} = \frac{\initial\lam}{ \widehat{\predicted (d)} }
\]
\vspace{0.25in}
	\item To approximate the push-forward of the initial density, we require:
\begin{assumption}\label{as:predx}
There exists some $C>0$ such that
\[
\observed (d) \leq C \widehat{\predicted(d)} \text{ for a.e. } d\in \dspace.
\]
\end{assumption}

\end{itemize}

\end{frame}


%%%%%%%%%%%%%%%%%%%%%%%%%%%%%%%%%%%%%%%%%%%%%%%%%%%%%%%%%%
\begin{frame}[t]
\begin{assumption}\label{as:predx}
There exists some $C>0$ such that
\[
\observed (d) \leq C \widehat{\predicted(d)} \text{ for a.e. } d\in \dspace.
\]
\end{assumption}

If this assumption is satisfied, we can prove the following cite{BJW18}:

\begin{theorem}
The error in the approximate updated is:
\begin{equation}\label{eq:pred_bound}
d_{\text{TV}} (\updatedP, \widehat{\updatedP}) \leq C d_{\text{TV}} (\predictedP, \widehat{\predictedP}),
\end{equation}
where the $C$ is the constant taken from \eqref{as:predx}.
\end{theorem}

\end{frame}


%%%%%%%%%%%%%%%%%%%%%%%%%%%%%%%%%%%%%%%%%%%%%%%%%%%%%%%%%%
\begin{frame}[t]{Practical Considerations}

\begin{itemize}
	\item We approximate $\predicted$ using density estimation on forward propagation of samples from $\initial$
	\item May evaluate $\updated$ directly for any sample of $\pspace$ (one model solve)
	\item Accuracy of the computed updated density is proportional to accuracy of approximation of the predicted density
	\item We (currently) use Gaussian KDE
	\begin{itemize}
			\item Let $\dimD$ be the dimension of $\dspace$
			\item Let $\nsamps$ is the number of samples from $\initial$ propagated through $\qoi$
			\item Converges at a rate of $\mathcal{O}(\nsamps^{-4/(4+\dimD)})$ in mean-squared error
			\item Converges at a rate of $\mathcal{O}(\nsamps^{-2/(4+\dimD)})$ in $L^1$-error
	\end{itemize}
\end{itemize}
\end{frame}
