We provide a summary of the notation, definitions, problem-formulation, and assumptions that reoccur throughout this work. 
For more details on the original sources and derivations,  we refer the interested reader to \cite{BES12, BE13, BET+14}. 

Let $u$ be the solution to a model $\M(u, \param) = 0$, perhaps of a physical system such as the amount of contaminant in a subsurface. 
Let $\param$ represent a parameter into such a model, e.g. the permeability of the medium in the subsurface through which a contaminant is spreading.
Such parameters are often uncertain and we begin the quantification of uncertainty by identifying the set of all physically plausible parameters denoted by $\pspace\subset\RR^\dimP$.
Since different choices of $\param \in \pspace$ often lead to different model solutions, we write $u\lam$ to make this dependence explicit.

However, we cannot in general observe the entire solution $u(\param)$.
Instead, we are often limited in our ability to observe data related to some quantities of interest (QoI), defined as functionals of $u\lam$ (e.g. the contaminant levels at a specific well at a particular time).
We let $\qoi$ denote the QoI map from the solution space of the model to the space of observable data. 
Then, given $\param \in \pspace$, we obtain $u\lam$ and compute $\qoi(u\lam)$ to get the QoI datum predicted by the model.
Clearly, the QoI map depends on $\param$ through the dependency of $u$ on $\param$, so we write $\qlam$ to simplify our notation.
We assume this map is at least piecewise-differentiable.	
The data space $\dspace \subset \RR^d$ is defined as the range of the QoI map $\qoi$, i.e. 
\[
\dspace = \qoi(\pspace).
\]
In other words, we use $\dspace$ to denote the space of all physically plausible data for the QoI that the model can predict.


Let $\pborel$ and $\dborel$ denote (the Borel) $\sigma$-algebras on $\pspace$ and $\dspace$, respectively.
The map $\qoi$ between measurable spaces $(\pspace, \pborel)$ and $(\dspace, \dborel)$ is immediately measurable by the smoothness assumption. 
Then, equipping $\pspace$ and $\dspace$ with (dominating) measures $\pmeas$ and $\dmeas$, respectively, is the final ingredient for defining probability density functions (pdfs) on the measure spaces $(\pspace, \pborel, \pmeas)$ and $(\dspace, \dborel, \dmeas)$.
In practice, $\pmeas$ and $\dmeas$ are often taken to be Lebesgue volume measures when $\pspace$ and $\dspace$ are finite-dimensional~\cite{BET+14, BJW18}.
In general, these measure allow for the description of probability measures as probability density functions defined by Radon-Nikodym derivatives.


\subsection{Problem Formulation and Solution}

We begin with defining the types of forward and inverse problems considered in this thesis.

\begin{defn}[Forward Problem]\label{defn:forward-problem}
  Given a probability measure $\PP_\pspace$ on $(\pspace, \pborel)$, and (at least piecewise-differentiable) QoI map $\qoi$, the \emph{forward problem} is to characterize a measure $\PP_\dspace$ on $(\dspace, \dborel)$ (recalling $\dspace = \qoi(\pspace)$) that satisfies
  \begin{equation}\label{eq:forward-problem}
    \PP_\dspace (E) = \PP_\pspace \left ( \qoi^{-1}(E) \right ), \; \forall \; E \in \dborel.
  \end{equation}
\end{defn}

\begin{defn}[Inverse Problem]\label{defn:inverse-problem}
  Given a probability measure $\observedP$ on $(\dspace, \dborel)$ that is absolutely continuous with respect to volume measure $\dmeas$, the \emph{inverse problem} is to determine a probability measure $\PP_\pspace$ on $(\pspace, \pborel)$, absolutely continuous with respect to $\pmeas$, satisfying

  \begin{equation}\label{eq:inverse-problem}
    \PP_\pspace (\qoi^{-1}(E)) = \int_{\qoi^{-1}(E)} \pp_\pspace \lam \, d\pmeas = \int_E \observed \q \, d\dmeas = \observedP(E), \; \forall \; E \in \mathcal{B}_\dspace.
  \end{equation} 

  \noindent Here,
   
  \begin{equation*}
    \pp_\pspace := \frac{d\PP_\pspace}{d\pmeas} \;\text{ and }\; \observed := \frac{d\observedP}{d\dmeas}
  \end{equation*}
  denote the Radon-Nikodym derivatives (i.e., pdfs) of $\updatedP$ and $\observedP$, respectively. 
  Any probability measure $\PP_\pspace$ satisfying \eqref{eq:inverse-problem} is referred to as a \emph{consistent solution} to the inverse problem, and \eqref{eq:inverse-problem} is referred to as the \emph{consistency condition}.
\end{defn}

In measure-theoretic terms, $\PP_\pspace$ is a pull-back measure of $\observedP$.
From the perspective of a forward problem, we seek $\PP_\pspace$ such that its \emph{push-forward measure is equivalent to} $\observedP$. 
In other words, the solution we seek to the inverse problem is constrained by a forward problem. 
Below, we formalize some of the vocabulary involved in the formulation and solution of the stochastic inverse problem.

\begin{defn}[Observed Distribution]\label{defn:observed}
  The density $\observed$ in \eqref{eq:inverse-problem} represents the uncertainty in QoI data and is referred to as the \emph{observed distribution} (or density), and is the Radon-Nikodym derivative of the \emph{observed measure} $\observedP$ with respect to the volume measure $\dmeas$.
\end{defn}

The map $\qoi$ impacts the structure of the update since the underlying data space $\dspace$ itself depends on $\qoi$, and both densities on $(\dspace, \dborel)$ are evaluated at $\qlam$.
In the event that the map $\qoi$ is a bijection, then the consistency condition \eqref{eq:inverse-problem} defines a unique measure $\PP$ on $\pspace$ given the specification of an observed density.
However, there are many applications of interest where $\qoi$ fails to be a bijection, either due to differences in the dimensions of the parameter and data spaces, nonlinearities inherent in the model itself, or both. 


The specific nuances of this relationship are discussed in \ref{chapter:02} and \ref{chapter:03} in greater detail.
Here, we simply note that the construction of \eqref{eq:update} requires only the forward-problem construction of $\predicted$, since $\initial$ and $\observed$ are given \emph{a priori}.
Additional properties are given in \ref{sec:properties} alongside the conditions for the existence and uniqueness of an update of the form given by \eqref{eq:inverse-problem}. 


