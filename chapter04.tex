\chapter{\uppercase{Data-Driven Maps and Consistent Inversion} \label{chapter:04}}

Going off the previous chapter, we may be interested in a different question, and may be in situations where we can collect data easily (or it has already been collected for us).
What are the other types of questions we can ask? Parameter Identification. 

We focus on extending the DCI framework to situations where we want to answer a different question.

\
\section{A Generalized Stochastic Map Framework}

Set stage with the work Troy has been doing lately.

\
\section{Data-Driven Maps}

More material from the paper, including sensitivity analysis with number of data points.

\
\section{Software Contributions}

Module into BET (started off as CBayes, make footnote or include in appendix) that transforms time-series into QoI and performs data-consistent inversion.

\
\section{Numerical Results and Analysis}

Take the exponential decay problem and show what happens as we increase the number of data points.

First, in fixed-time window (say, [1,2]), then at fixed frequency, go up to 100 measurements. Show results.

Show a problem with repeated observations (so, slightly different context, but similar question, and now in a position to compare the measure and density approaches). Use example that was presented at SIAM AN18, except re-formulated as measuring the weight and volume of a liquid. 