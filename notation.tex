\chapter*{\uppercase{Abbreviations and Notation} \label{appendix-notation}}

\section{Abbreviations}

\begin{description}[leftmargin=!,labelwidth=0.5in,font=\normalfont]
	\item 
\end{description}


\section{Mathematical Notation}
\begin{description}[leftmargin=!,labelwidth=0.5in,font=\normalfont]
	\item 
\end{description}


\subsection{General Notation}

% EDIT BELOW AND REMOVE WHAT YOU DO NOT NEED
\begin{description}[leftmargin=!, labelwidth=0.7in]
  \item[$x$]             italicized, Roman or Greek letter, denotes a scalar values 
  \item[$\bs{x}$]        italicized, bold, lowercase Roman or Greek letter, denotes a column vector or set.
  \item[$\bs{X}$]        italicized, bold, uppercase Roman or Greek letter, denotes a matrix or set.
  \item[$\card{\bs{x}}$] cardinality, number of elements, of the vector or set $\bs{x}$ 
  \item[$x \in (a, b)$]  the value $x$ is within the interval such that $a < x < b$
  \item[{$x \in (a, b]$}]  the value $x$ is within the interval such that $a < x \leq b$
  \item[$x \in [a, b)$]  the value $x$ is within the interval such that $a \leq x < b$
  \item[{$x \in [a, b]$}]  the value $x$ is within the interval such that $a \leq x \leq b$ 
  \item[$1_{A}\left(x\right)$] the indicator function,\[1_{A}\left(x \right) = \begin{cases} 1 & x \in A \\ 0 & x \notin A \end{cases}.\] 
  \item[$\bs{1}_n$] an column vector of $n$ $1$s
  \item[$\bs{I}$] the identity matrix
  \item[$\bs{I}_n$] the $n \times n$ identity matrix
  \item[$\bs{X}^{-1}$] the inverse matrix, i.e., the operator satisfying $\bs{X}^{-1} \bs{X} = \bs{X} \bs{X}^{-1} = \bs{I}.$
  \item[$\bs{X}^{T}$] transpose, i.e. the operator satisfying $T_{i,j} = T_{j, i} \; \forall \; i,j \in \mathbb{N}$
  \item[$\otimes$] Kronecker product
  \item[$\odot$] element-wise multiplication
\end{description}

\subsection{Sets}

\begin{description}[leftmargin=!, labelwidth=0.8in]
  \item[$\{x, y, z, \ldots\}$]  The set comprising the elements of $x,$ $y,$ $z,$ $\ldots$ 
  \item[$\{x, y, z\} \backslash x$]  The set comprising the elements of $y$ and $z,$ that is, the backslash removes elements from the set.
  \item[$\left\{x_i\right\}_{i = 1}^{n}$]  The set comprising the elements of $x_1, x_2, x_3, \ldots, x_n.$ 
  \item[$\mathbb{R}$] Set of real numbers 
  \item[$\bs{x} \in \mathbb{R}^n$] $\bs{x}$ is a vector with $n$ elements, all of which are real numbers.
\end{description}

\subsection{Statistical Distributions}
\begin{description}[leftmargin=!, labelwidth=0.8in]
  \item[$\mathcal{N} \left( \mu, \sigma^2 \right)$] the uni-variable Gaussian distribution with mean $\mu$ and variance $\sigma^2.$ 
  \item[$\mathcal{N} \left( \bs{\mu}, \bs{\Sigma} \right)$] the multi-variable Gaussian distribution with mean vector $\bs{\mu}$ and variance-covariance matrix $\bs{\Sigma}.$
  \item[$\phi(x)$] The standard Gaussian density function.
\end{description}

\subsection{Specialized Notation}
This notation refers to values defined explicitly in the context of this thesis.

\begin{itemize}
\itembox{$\M$} A model that takes input parameters to an (observable) state space.
\itembox{$u$} An observable state space from which data is to be collected.
\itembox{$\param$} A (model) parameter into model $\M$.
\itembox{$\obs$} A Parameter-to-Observables (PtO) map, denoted $\obs( u\lam )$ or $\obs\lam$, each component of which is a functional on the observable state, $\obs_i: u\lam \to \RR$. This map represents performing an individual experiment, which may consist of one or more observations (either in space or time). Arranging observational data into a vector defines the map.
\itembox{$\qoi$} A Quantity of Interest (QoI) map, denoted $\qoi$, which acts on observable data to transform the data into a scalar or vector quantity. For example, it may be the average of measurements encompassed in $\obs$. Technically, this map encompasses the following set of compositions: $Q(\obs(u\lam))$.
\itembox{$\data$} Data. This symbol is used to represent the output of a QoI map, i.e. $\qoi = \data$.
\end{itemize}