%%%% 3D Skewness Example %%%%%%

\subsection{Dependence on Dimension}\label{ex:3dmap}
We extend the numerical investigation to a parameter space of dimension three to further illustrate that these results hold as we move towards higher dimensions.
Generally, we have fewer QoI than number of uncertain model parameters, so we assume that the potential QoI maps are defined by the $2\times 3$ matrices
\begin{equation}\label{eq:qmap3}
\qspace_S := \left \lbrace \qoi^{(s)} =  \mat{ccc}{1 & 0 & 0\\ \sqrt{s^2 - 1}& 1 & 0} \right \rbrace_{s\in S}.
\end{equation}
Here, as in the previous example, the index $s$ indicates the magnitude of skewness.
Furthermore, the results of Example~\ref{ex:rotation} justify the restriction of the maps to this form since any linear map of skewness $s$ is simply a rotation of maps of this form.

%Now, the generalized contours for inverses of maps from $\RR^3 \to \RR^2$ will be isomorphic to 2\--dimensional contour events in that the inverse sets will be columns in 3\-space orthogonal to the aforemntioned plane.
%
%IMAGE DEMONSTRATING THIS WOULD HELP.
%We define
%
%which is just the map from \eqref{eq:qmap2} appended with zeros in the third column.
%We make this choice solely for convience and are justified in doing so owing to Proposition~\ref{prop:rot_invariance} and the fact of generalized contours of maps from $\RR^3 \to \RR^2$ being parallel columns.
%The rotational invariance naturally extends to the third dimension.

%We note that $\bar{N}$ is much higher since we kept the convention of 200 grid cells per dimension in our reference.
%However, we kept the same number of random samples $N$, so we should expect higher errors due to the overresolved regular grid.
%Fortunately, we find that the results still generalize.
%We present the case where $M=1$:

\begin{figure}[h]
\begin{minipage}{.5\textwidth}
\begin{table}[H]
\begin{tabular}{ c | c | c | c }
\nsamps & $\qoiA$ & $\qoiB$ & $\qoiC$\\ \hline \hline
$200$ & $2.98E-01$ & $4.18E-01$ & $5.60E-01$\\ \hline

$400$ & $2.27E-01$ & $3.27E-01$ & $4.69E-01$\\ \hline

$800$ & $1.81E-01$ & $2.70E-01$ & $3.97E-01$\\ \hline

$1600$ & $1.46E-01$ & $2.15E-01$ & $3.09E-01$\\ \hline

$3200$ & $1.15E-01$ & $1.72E-01$ & $2.44E-01$\\ \hline

$6400$ & $9.09E-02$ & $1.39E-01$ & $1.95E-01$\\ \hline
\end{tabular}
\end{table}
\end{minipage}
\begin{minipage}{.45\textwidth}
		\includegraphics[width=\linewidth]{./images/Plot-reg_BigN_8000000_reg_M_1_rand_I_100000}
\end{minipage}
\caption{The results of $d^2_H(\PP_{\pspace, \ndiscs, \nsamps}, \PP_{\pspace, \ndiscs, \bar{\nsamps}})$ for $\ndiscs = 1, \bar{\nsamps} = 8,000,000$, with $a, b, c = 1, 2, 4$.}
\label{fig:M1_3d}
\end{figure}
\FloatBarrier
In Figure~\ref{fig:M1_3d}, it appears that the effect of skewness is even more pronounced in higher dimensions, and that the number of samples required to achieve similar levels of accuracy between two maps with a ratio of skewness 2 is now quadrupled.
The analysis of \cite{BGE+15} suggested a dependence of accuracy related to the skewness raised to a power related to the dimension of the data space.
