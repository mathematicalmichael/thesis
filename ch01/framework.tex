We provide a summary of the notation, definitions, problem-formulation, and assumptions that reoccur throughout this work. 
For more details on the original sources and derivations,  we refer the interested reader to \cite{BES12, BE13, BET+14}. 

Let $u$ be the solution to a model, perhaps of a physical system such as the amount of contaminant in a subsurface. 
Let $\param$ represent a parameter into such a model, e.g. the permeability of the medium in the subsurface through which a contaminant is spreading.
Such parameters are often uncertain and we begin the quantification of uncertainty by identifying a set of possible parameters $\pspace$.
Since different $\param \in \pspace$ can lead to different model solutions, we write $u\lam$ to make this dependence explicit.

Generally, we cannot observe the entire solution $u(\param)$, and instead are able to observe data related to some quantities of interest (QoI), defined as functionals of $u\lam$ (e.g. the contaminant levels at a specific well at a particular time).
We let $\qoi$ denote the QoI map from the solution space of the model to the space of observable data. 
Then, given $\param \in \pspace$, we obtain $u\lam$ and compute $\qoi(u\lam)$ to get the QoI data predicted by the model.
Clearly, the QoI map depends on $\param$ through the dependency of $u$ on $\param$, so we write $\qlam$ to simplify our notation.
We assume this map is at least piecewise-differentiable.	
The data space $\dspace \subset \RR^d$ is defined as the range of the QoI map $\qoi$, i.e. 
\[
\dspace = \qoi(\pspace).
\]

Let $\pborel$ and $\dborel$ denote (the Borel) $\sigma$-algebras on $\pspace$ and $\dspace$, respectively.
The map $\qoi$ between measurable spaces $(\pspace, \pborel)$ and $(\dspace, \dborel)$ is immediately measurable by the smoothness assumption. 
Then, equipping $\pspace$ and $\dspace$ with (volume) measures $\pmeas$ and $\dmeas$, respectively, is the final ingredient for defining probability density functions (pdfs) on the measure spaces $(\pspace, \pborel, \pmeas)$ and $(\dspace, \dborel, \dmeas)$.
In practice, $\pmeas$ and $\dmeas$ are often taken to be Lebesgue volume measures when $\pspace$ and $\dspace$ are finite-dimensional~\cite{BET+14, BJW18}.

%%%%%%%%%%% Section 2.1
\subsection{Problem Formulation and Solution}
This now brings us to our central definition:

\begin{defn}[Inverse Problem]\label{defn:consistency}
Given a probability measure $\observedP$ on $(\dspace, \dborel)$ that is absolutely continuous with respect to $\dmeas$, the \emph{inverse problem} is to determine a probability measure $\PP_\pspace$ on $(\pspace, \pborel)$, absolutely continuous with respect to $\pmeas$, such that

\begin{equation}\label{eq:inverse-problem}
\PP_\pspace(\qoi^{-1}(E)) = \int_{\qoi^{-1}(E)} \pp\lam \, d\pmeas = \int_E \observed \q \, d\dmeas = \observed(E), \; \forall \; E \in \mathcal{B}_\dspace.
\end{equation} 

Here, 
\begin{equation*}
\updated = \frac{d\PP_\pspace}{d\pmeas} \;\text{ and }\; \observed = \frac{d\observedP}{d\dmeas}
\end{equation*}
are the Radon-Nikodym derivatives (i.e., pdfs) of $\PP_\pspace$ and $\observed$, respectively. 

Any probability measure $\PP_\pspace$ satisfying \eqref{eq:inverse-problem} is referred to as a \emph{consistent solution} to the inverse problem, and \eqref{eq:inverse-problem} is referred to as the \emph{consistency condition}.
\end{defn}

In measure-theoretic terms, $\PP_\pspace$ is a pull-back measure of $\observedP$.
From the perspective of a forward problem, we seek $\PP_\pspace$ such that its \emph{push-forward measure is equivalent to} $\observedP$. 
In other words, the solution we seek to the inverse problem is constrained by a forward problem. 
Below, we introduce some vocabulary involved in the formulation and solution of the stochastic inverse problem.

\begin{defn}[Observed Density]\label{defn:obsden}
The density $\observed$ in \eqref{eq:inverse-problem} represents the uncertainty in QoI data and is referred to as the \emph{observed density}.
\end{defn}

To construct $\updated$ satisfying \eqref{eq:inverse-problem}, we adopt a Bayesian perspective of incorporating prior beliefs and data. 

\begin{defn}[Initial Distribution]\label{defn:initial}
The density $\initial$ is used to represent any prior beliefs about (or \emph{initial descriptions of}) parameters before evidence is taken into account, and is referred to as the \emph{initial density}.
\end{defn}

To construct $\updated$, we ``push-forward'' the initial beliefs using the QoI map to compare to the evidence provided by $\observed$. 
In other words, we first solve a forward problem to construct a solution to the inverse problem. 
To make this precise, we use the following:

\begin{defn}[Predicted Distribution]\label{defn:predicted}
The push-forward density of $\initial$ under the map $\qoi$ is referred to as the \emph{predicted} distribution, denoted by $\predicted^Q$, or $\predicted$ when there is no ambiguity about the map under consideration. 
It is given as the Radon-Nikodym derivative (with respect to $\dmeas$) of the push-forward probability measure defined by 
\begin{equation}\label{eq:updated-measure}
\predictedP^Q (E)  = P_\pspace \left ( \qoi^{-1}(E) \right ), \; \forall \; E \in \dborel.
\end{equation}
\end{defn}

These definitions are combined to form the \emph{updated density}, originally derived in \cite{BJW18}, visible in \ref{eq:consistent-solution}, and summarized as follows:
\begin{defn}[Updated Distribution]\label{defn:updated}
\begin{equation}\label{eq:updated}
\updated \lam := \initial \lam \frac{\observed \q }{\predicted \q }, \; \param \in \pspace.
\end{equation}
\end{defn}

Any probability measure that satisfies Equation~\eqref{eq:inverse-problem} is considered a consistent solution to the stochastic inverse problem.

In the event that the map $Q$ is a bijection, then the consistency condition (\ref{defn:consistency}) defines a unique measure $\PP$ on $\pspace$ given the specification of an observed density.
However, there are many applications of interest where the quantity of interest map $\qoi$ fails to be a bijection, either due to differences in the dimensions of the parameter and data spaces, nonlinearities inherent in the model itself, or both. 

The map $\qoi$ impacts the structure of the update, since both densities on $(\dspace, \dborel)$ are evaluated at $\qlam$.
Moreover, the underlying data space $\dspace$ itself depends on $\qoi$.
The nuances of this relationship are discussed in \ref{chapter:02} and \ref{chapter:03} in greater detail.
Here, we simply note that the construction of \eqref{eq:update} requires only the forward-problem construction of $\predicted$, since $\initial$ and $\observed$ are given \emph{a priori}.
Some additional properties are summarized below along with some critical assumptions necessary to guarantee the existence and uniqueness of a update of the form given by \eqref{eq:inverse-problem}. 