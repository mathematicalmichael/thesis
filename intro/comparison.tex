%%%%%%%%%%%%%%%%%%%%%%%%%%%%%%%%%%%%%%%%%%%%%%%%%%%%%%%%%%%%%%%%%%%%
%%%%%%%%%%%%%%%%%%%%%%%%%%%%%%%%%%%%%%%%%%%%%%%%%%%%%%%%%%%%%%%%%%%%
\section{Comparing Inverse Problems and Solutions}\label{sec:compare}
%%%%%%%%%%%%%%%%%%%%%%%%%%%%%%%%%%%%%%%%%%%%%%%%%%%%%%%%%%%%%%%%%%%%
%%%%%%%%%%%%%%%%%%%%%%%%%%%%%%%%%%%%%%%%%%%%%%%%%%%%%%%%%%%%%%%%%%%%

% The Bayesian framework poses a different question and seeks a different answer than the Data-Consistent one.
The objective of the Bayesian approach is to determine a single ``true'' parameter that explains all of the observed data \citep{Smith, Concrete, Complete, Stuart10, Tarantola_book}.
The philosophical underpinnings of Bayesian inference is akin to asking the following:

\begin{center}
  \emph{How does one incorporate collected data to shift prior beliefs about specific parameter values?}
\end{center}

% The motivations of Bayesian inference are distinct from the Data-Consistent framework, where we seek a pull-back measure: a description of the uncertainty set that explains the variation in the observations under a given description of error.
However, the Data-Consistent approach reflects the goal of constructing a distribution (or probability measure), which is akin to asking:

\begin{center}
  \emph{How does one update initial beliefs in such a way that the modified predictions match the description of uncertainty in observed data?}
\end{center}

We now develop a typical Bayesian inverse problem following the framework described in \cite{Stuart10, Smith}, in order to make a formal comparison to the Data-Consistent SIP.

\subsubsection{The Bayesian Inverse Problem}
Let $d$ denote the ``noisy'' data obtained on $Q(\paramref)$, which is often represented as
\begin{equation*}
	d = Q(\paramref) + \xi,
\end{equation*}
where $\xi$ is a random variable used to model the measurement error that is often assumed to follow a Gaussian distribution.
Then, the data-likelihood function, often written as a conditional density, $L_\dspace(\q \, |\, \param)$, is formed.
This describes the differences in relative likelihoods that the data could have been generated from a particular $\param$.
Ideally, the largest values of $L_\dspace(\q \, | \, \param)$ occur whenever $\param$ is a close approximation of the true parameter $\paramref$.
The data-likelihood function is distinct from the observed density used in the data-consistent framework.

The next step in constructing a Bayesian inverse problem is the specification of a prior density denoted by $\pi_\text{prior}(\param)$.
The prior describes the different relative likelihoods assumed for the true parameter before data are collected.
This is also distinct from the role of the {\em initial} density used in the data-consistent framework.
% We choose them to represent the set of feasible parameters, and rely on Monte-Carlo sampling for both approaches\footnote{Priors in Bayesian inference are sometimes chosen for reasons related to Markov-Chain Monte-Carlo algorithms in order to ensure their balancing of investigation and exploration, or convergence [TK - cite someone]}.

The posterior density (i.e., the formal solution to the Bayesian inverse problem) is given by a conditional density, denoted by $\pi_\text{post}(\param\, | \, d)$, proportional to the product of the prior and data-likelihood function \citep{Walpole, Berger, Complete, Smith}.
In other words,
\begin{equation*}
	\posterior(\param\, | \, \q) \propto \prior(\param)L_\dspace(\q\, | \, \param)
\end{equation*}
This form of the density follows from Bayes' rule (not from the Disintegration Theorem as with the updated density).
The posterior can be interrogated to assess the difference in relative likelihoods of a fixed parameter given the observed data.
Subsequently, the posterior is often used to produce a ``best'' estimate of the true parameter.
For example, the maximum a posteriori (MAP) point is the parameter that maximizes the posterior density.

Unlike the updated density, the posterior is often interrogated indirectly without knowledge of the normalizing constant $C$ required to represent it as a density.
In other words, the posterior is a density of the form

\begin{equation}\label{eq:sb_post}
    \posterior\lam := \prior\lam \frac{L_\dspace (\q | \param)}{ C },
\end{equation}

\noindent where we emphasize the use of $\posterior$ to distinguish the \emph{posterior} from the updated density $\updated$ in \eqref{eq:updated-pdf}.
Here, the denominator $C$ (known as the \emph{evidence} \cite{Smith}), ensures the posterior density integrates to one, and is given by
\[
C = \int_\pspace \prior\lam L_\dspace(\q | \param) \, d\param.
\]

Note that there are no constraints or requirements that likelihood function be a density.
In fact, $L_\dspace$ need not even be in $L^1(\pspace)$ since it is actually only the product $\prior(\param) L_\dspace (\q | \param)$ that is required to be in $L^1(\pspace)$ to form a posterior.
In other words, $L_\dspace (\q | \param)$ and $\observed(\qlam)$ can model completely different things with respect to uncertainty in the data.
As stated above, $L_\dspace (\q | \param)$ is used to assess the relative likelihood that a single parameter $\param\in\pspace$ explains all of the observed data, whereas $\observed(\qlam)$ describes the relative likelihood of a predicted datum associated with $\param\in\pspace$.
In the Bayesian framework, there is a different notion of consistency of solutions, referring to certain asymptotic properties of $\posterior$ in the limit of infinite data \citep{Barron, Silverman, starktenorio}.



%%%%%%%%%

\subsection{Comparison of Data-Consistent and Bayesian Inversion}\label{sec:bayesian}

We summarize the posterior and updated densities side-by-side in Table~\ref{tab:dens_comparisons} and comment on a few notable aspects not mentioned above.
Observe for the posterior density that the data-likelihood function appears in both the numerator and denominator.
In particular, the data-likelihood function informs the {normalizing constant}, commonly referred to as the evidence term, in the denominator.
This is in contrast to the denominator of the updated density, which is given by the predicted density, which is in general not a constant, and can be constructed independently of $\observed$.

\begin{table}[htbp]
\centering
\begin{tabular}{|c|c|}
\hline
 & \\
$\displaystyle \updated(\param) = \initial(\param) \frac{\observed(\q)}{\predicted(\q)}
$
&
$
	\displaystyle \pi_{\text{post}}(\param\,|\,\q) = \frac{\prior(\param)L_\dspace(\q \,|\,\param)}{\int_{\Lambda} L_\dspace(\q \, |\, \param)  \prior(\param) \, d\pmeas}
$
 \\ & \\ \hline
\end{tabular}
\caption{Updated density solving the SIP (left) and posterior density solving the Bayesian inverse problem (right).}
		\label{tab:dens_comparisons}
\end{table}

A practical implication of this difference is that the updated density only alters the structure of the initial density in what we refer to as the ``data-informed'' parameter directions.
Specifically, for a fixed $\q\in\dspace$, let $C_\q := \set{\param\in\pspace\, : \, \qlam=\q}$, i.e., $C_\q$ is a ``contour'' in parameter space.
Then, for any $\param\in C_\q$, we immediately have $\updated(\param)=r(\q)\initial(\param)$ where $r(\q)$ is a fixed constant of proportionality for all $\param\in C_\q$.
%Subsequently, using the Disintegration theorem on both the initial and updated densities produces exactly the same family of conditional densities on the contours in parameter space.
By contrast, while the posterior does not have to agree with the prior in any direction in parameter space, the prior does impact the structure of the posterior in all directions.

The previous paragraph is not\---and should not be interpreted as\---a criticism of the Bayesian inverse framework.
It is simply meant to highlight that the data-consistent and Bayesian frameworks formulate and solve inverse UQ problems from different perspectives and with different (although at times seemingly compatible) assumptions.
Consequently, the solutions for an inverse problem formulated under either framework may differ significantly.
As the example (adopted from \cite{BJW18a, BJW18b}) below demonstrates, this is true even if we arbitrarily force the inverse problems to appear as similar as possible.

\FloatBarrier
