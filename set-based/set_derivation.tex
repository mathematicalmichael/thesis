To properly summarize the set-based solution, we define several measure/probability spaces and refer to the schematic given in Figure \ref{fig:scheme} in order to illustrate the steps and spaces required in the formulation and solution of the SIP.
For a more extensive review, we refer the reader to \cite{BET+14, MBD+15, BBE11, BES12, BET+14}, and \cite{BGE+15}.

%%%%%%%%%%%%%%%%%%%%%%%%
\begin{figure}[!h]
\begin{equation}
\underbrace{
\underbrace{
\overbrace{
 \Pspace \xmapsto{\  \qoi \ } \Dspace
  \xmapsto{\ \observedP \ } \Ospace
 }^{
 \text{(S1): Stochastic Inverse Problem (SIP)}
 }
 \xmapsto{\ \qoi^{-1} \ } (\pspace, \cborel, \contourP)
 }_{
 \text{(S2): Solution to SIP Satisfying Eq. \eqref{eq:dataspace_pushforward_measure}}}
 \xmapsto{\ \set{\PP_\ell}_{\ell\in\mathcal{L}} \ } (\pspace, \pborel, \paramP)
 }
 _{
 \text{(S3): Unique Solution to SIP by Eq.~\eqref{eq:disintegration_measure} and Ansatz}
 }
\end{equation}
\caption{The first step (S1) defines (i)~the formulation of the SIP by specification of the model, (ii)~the measure spaces of parameters and (iii)~observable outputs, and (iv)~the probability measure on the latter. The second step (S2) defines a unique solution to the SIP on the space $\pspace$ equipped with the contour $\sa$ $\cborel$ using the definition of the push-forward measure. In (S3), the Disintegration Theorem and and Ansatz are applied to define a unique solution on the space of interest $(\pspace, \pborel)$ equipped with a probability measure $\paramP$.}
\label{fig:scheme}
\end{figure}


The initial measure/probability spaces involved in the formulation of the SIP are summarized in step (S1) of Fig.~\ref{fig:scheme}, starting with measure space $\Pspace$.
The assumption that $\qoi$ is at least piecewise-differentiable implies the measurability of the QoI map, so that the space $\dspace$ induced by $\qoi$ is equipped with the Borel $\sa$ $\dborel$ \citep{Hunter}.
The push-forward measure $\dmeas$ on ${(\dspace, \dborel)}$ is defined as

\begin{equation}\label{eq:dataspace_pushforward_measure}
\dmeas (A) = \int_A \, d\dmeas := \int_{\qoi^{-1}(A)} \, d\pmeas = \pmeas \left (\qoi^{-1}(A) \right ) \quad \forall \;  A\in\dborel,
\end{equation}

\noindent which defines the measure space $\Dspace$\footnote{When referring to properties of the data space that are not unique to the choice of map used to induce $\dspace$, we will drop the subscript notation and assume the dependence is understood, as expressed in Fig.~\ref{fig:scheme}.}.
In practice, when $\dmeas$ is absolutely continuous with respect to the $\dimD$--dimensional Lebesgue measure, we substitute the Lebesgue measure for $\dmeas$.

The final step in (S1) involves the specification of a probability measure $\dataP$ on ${(\dspace, \dborel)}$ to model the uncertainty in data.
This leads to the SIP from Def.~\eqref{defn:inverse-problem}: determine a probability measure $\paramP$ on ${(\pspace, \pborel)}$ such that

\begin{equation}\label{eq:inverse_measure}
\paramP \left ( \qoi^{-1}(E)\right ) = \dataP(E) \; \forall \; E \in \dborel.
\end{equation}

\noindent This equation implies that any solution is uniquely determined on the induced contour $\sa$
\begin{equation}\label{eq:contour_sa}
\cborel = \set{\qoi^{-1}(E) : E \in \dborel } \subset \pborel,
\end{equation}
which is summarized as step (S2) of Fig.~\ref{fig:scheme}.
However, for sets $A \in \pborel \setminus \cborel$, more information is required than is provided in Eq.~\eqref{eq:inverse_measure} in order to determine $\paramP (A)$.
When solutions to the SIP are given by densities, we form a family of conditional densities using an initial density.
In the set-based approach, we do not assume an initial density on $(\pspace, \pborel)$.
Instead, we consider approximations of contour events.
Below, we describe these structures and the relationship to the Disintegration theorem.

By the Implicit Function Theorem, if $\qlam \in C^1 (\pspace)$ and we let $\data\in\dspace$ be a fixed datum, $\qoi^{-1}(q)$ exists as a $(\nparams-\ndata)$\--dimensional manifold (possibly piecewise-defined) that we refer to as a \emph{generalized contour} \cite{BET+14}.
These generalized contours can be indexed by a $\dimD$--dimensional manifold (also possibly piecewise-defined) of dimension $\ndata$ called a \emph{transverse parameterization} that intersects each contour once and only once.
In \cite{BET+14}, it is shown that transverse parameterizations are guaranteed to exist and can be approximated by a finite number of $\dimD$---dimensional hyperplanes when $\pspace$ is compact.
In general, the transverse parameterization is not unique.

We let $\LL$ denote any particular transverse parameterization.
Each $\ell\in\LL$ corresponds to a unique generalized contour $\CC_\ell \in \pspace$ and each point $\param\in\pspace$ belongs to a unique $\CC_\ell\in\pspace$.
Thus, a transverse parameterization defines a bijection between the manifold $\LL$ and the partitioning of $\pspace$ into generalized contours that decomposes $\pspace$ in terms of equivalence classes.
The induced $\sa$ $\cborel$ and this bijection can then be used to define the measurable space $(\LL, \BB_\LL)$.

We denote the projection map $P_\LL : \pspace \to \LL$, and let $\set{\CC_\ell}_{\ell\in\LL}$ represent the family of generalized contours indexed by $\LL$, yielding the associated family of measurable spaces $\set{\left ( \CC_\ell, \BB_{\CC_\ell} \right )}_{\ell\in\LL}{}$.
By the Disintegration Theorem \citep{BES12, Dellacherie_Meyer_book}, any $\paramP$ is now defined completely in terms of structures embedded in $(\pspace, \pborel)$ as a (marginal) probability measure $\PP_\LL$ on $(\LL, \BB_\LL)$ and a family of (conditional) probability measures $\set{\PP_\ell}_{\ell\in\LL}$ on $\set{\left ( \CC_\ell, \BB_{\CC_\ell} \right )}_{\ell\in\LL}$ such that
\begin{equation}\label{eq:disintegration_measure}
\paramP (A) = \int_{P_\LL(A)} \left ( \int_{P_{\LL}^{-1} (\ell) \cap A}\, d\PP_\ell(\param) \right )\, d\PP_\LL (\ell), \; \forall \; A \in \pborel
\end{equation}

\noindent The disintegration of Eq.~\eqref{eq:disintegration_measure} implies that a specification of a family of conditional probability measures $\set{P_\ell}_{\ell\in\LL}$ gives us a unique solution to the SIP on ${(\pspace, \pborel)}$ since the marginal $\PP_\LL$ on ${(\LL, \BB_\LL)}$ is uniquely determined by $\observedP$ on ${(\dspace, \dborel)}$.

The conditional measures are not determined by the specification of $\dataP$.
We follow the work of \cite{BET+14} and adopt the \emph{standard ansatz} determined by the disintegration of the measure $\pmeas$ to compute probabilities of sets contained within contour events whenever $\pmeas(\pspace) < \infty$, e.g. when $\pmeas$ is the $\dimP$--dimensional Lebesgue measure and $\pspace \in \RP$ is precompact.
The standard ansatz is given by

\begin{equation}\label{eq:standard_ansatz}
\PP_\ell = \mu_{\CC_\ell} / \mu_{\CC_\ell}(\CC_\ell), \; \forall \; \ell \in \LL,
\end{equation}

\noindent where $\mu_{\CC_\ell}$ is the disintegrated volume measure on generalized contour $\CC_\ell$.
Thus, we have defined a unique solution to the SIP on ${(\pspace, \pborel)}$, completing step (S3) in Fig.~\ref{fig:scheme}.

In the absence of other information about differences in relative likelihoods of parameters, the standard ansatz effectively implies a uniform distribution describing the initial state of uncertainty about the input parameters\footnote{In the event that $\pspace$ is compact.}.
In the vocabulary of the density-based approach, the measure $\paramP$ can be viewed as updating an initial uniform measure on $\pspace$ in directions informed by the Quantity of Interest map, given uncertain data characterized by $\dataP$.

However, unlike the density-based approach that utilizes the pushforward of an initial density in the construction of a solution, the ansatz is imposed only on contours in $\pspace$.
Moreover, this does not assume any absolute continuity of probability measures.
We next turn our attention to the numerical approximation of the solutions to the SIP that follow from the set-based method.

\FloatBarrier
