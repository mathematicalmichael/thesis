\subsection{Descriptions of Error}\label{sec:set-error}
If we assume $\observed$ is absolutely continuous with respect to $\dmeas$, we can describe $\observed$ with a density $\observedP$. Then, for any partition $\set{D_\idisc}_{\idisc=1}^{\ndiscs}$ of $\dspace$,
\[
\observed (D_\idisc) = \int_{D_\idisc} \observed \, \dmeas, \quad \text{ for } \idisc = 1, \hdots, \ndiscs.
\]

We often use Monte Carlo approximations to compute the approximations $p_{\dspace, \idisc}=\observed(D_\idisc)$ in the first for-loop in Algorithm~\ref{alg:inv_density}.
These samples are generated on $\dspace$ and do not require numerical solutions to the model.
We therefore assume that for any discretization of $\dspace$, these approximations can be made sufficiently accurate and neglect the error in this computation.

We denote the exact solution to the SIP associated with this partitioning of $\dspace$ by $\PP_{\pspace, \ndiscs}$.
In situations where $\qoi(\param^{(\iparam)})$ is estimated (e.g. by application of a functional on a finite-element solution to a PDE), the approximate solutions to the SIP given in the final for-loop of Algorithm~\ref{alg:inv_density} are denoted by $\PP_{\pspace, \ndiscs, \nsamps, h}$.
Here, the $h$ is in reference to a mesh or other numerical parameter that determines the accuracy of the numerical solution $u_h(\param^{(\iparam)})\approx u(\param^{(\iparam)})$, and subsequently the accuracy in the computations of $\qoi_\iparam = \qoi(\param^{(\iparam)})$ in Algorithm~\ref{alg:inv_density}.
Then, by repeated application of the triangle inequality,
\begin{equation}
\label{eq:set-triangleineq}
d(\PP_{\pspace, \ndiscs, \nsamps, h}, \paramP) \leq
\underset{ \text{(E1)} }{\underbrace{d(\PP_{\pspace, \ndiscs, \nsamps, h},\PP_{\pspace, \ndiscs, \nsamps})}} +
\underset{ \text{(E2)} }{\underbrace{d(\PP_{\pspace, \ndiscs, \nsamps}, \PP_{\pspace, \ndiscs}) }}+
\underset{ \text{(E3)} }{\underbrace{d(\PP_{\pspace, \ndiscs}, \paramP) }}.
\end{equation}

The term (E1) describes the effect of the error in the numerically evaluated $\qoi_\iparam$ on the solution to the SIP.
The term (E2) describes the effect of finite sampling error in $\pspace$ on the solution to the SIP and (E3) describes the effect of discretization error of $\observed$ on the solution to the SIP.

We assume that $h$ is tunable so that for any $A\in \pborel$,
\[
\lim\limits_{h \downarrow 0} \PP_{\pspace, \ndiscs, \nsamps, \imesh} (A) = \PP_{\pspace, \ndiscs, \nsamps} (A).
\]
It is possible to prove the convergence of $\PP_{\pspace, \ndiscs, \nsamps, \imesh} (A) \to \paramP (A)$ for some $A\in \pborel$ and on estimating the error in $\PP_{\pspace, \ndiscs, \nsamps, h}(A)$.
For example, in \cite{BGE+15}, adjoint-based a posteriori estimates in the computed QoI are combined with a statistical analysis to both estimate and bound the error in $\PP_{\pspace, \ndiscs, \nsamps, \imesh} (A)$.
In \cite{JNME19}, adjoints are used to compute both error and derivative estimates of $\qoi(\param^{(\iparam)})$ to improve the accuracy in $\PP_{\pspace, \ndiscs, \nsamps, \imesh} (A)$.
Since the error due to $\imesh$ can be estimated as described in previous studies, and $\ndiscs$ can be made arbitrarily large, we neglect (E1) and (E3) here.
Thus, we limit our focus to (E2), where certain geometric properties of the QoI map (namely, skewness), are known to significantly impact this term.

\FloatBarrier
