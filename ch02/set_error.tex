\subsection{Descriptions of Error}\label{sec:set-error}
Recall that we assumed $\dataP$ is absolutely continuous with respect to $\dmeas$, which allows us to describe $\dataP$ with a density $\rho_\dspace$. Then, for any partition $\set{D_\idisc}_{\idisc=1}^{\ndiscs}$ of $\dspace$,
\[
\dataP (D_\idisc) = \int_{D_\idisc} \rho_\dspace \, \dmeas, \quad \text{ for } \idisc = 1, \hdots, \ndiscs.
\]

We often use Monte Carlo approximations to compute the approximations $p_{\dspace, \idisc}=\dataP(D_\idisc)$ in the first for-loop in Algorithm~\ref{alg:inv_density}.
These samples are generated on $\dspace$ and do not require numerical solutions to the model.
We therefore assume that for any discretization of $\dspace$, these approximations can be made sufficiently accurate and neglect the error in this computation.

We denote the exact solution to the SIP associated with this partitioning of $\dspace$ by $\PP_{\pspace, \ndiscs}$.
Approximate solutions to the SIP given in the final for-loop of Algorithm~\ref{alg:inv_density} are denoted by $\PP_{\pspace, \ndiscs, \nsamps, h}$.
Here, the $h$ is in reference to a mesh or other numerical parameter that determines the accuracy of the numerical solution $u_h(\param^{(\iparam)})\approx u(\param^{(\iparam)})$, and subsequently the accuracy in the computations of $\qoi_\iparam = \qoi(\param^{(\iparam)})$ in Algorithm~\ref{alg:inv_density}.

We assume that $h$ is tunable so that
\[
\lim\limits_{h \downarrow 0} \PP_{\pspace, \ndiscs, \nsamps, \imesh} = \PP_{\pspace, \ndiscs, \nsamps}.
\]
In \cite{BM17}, the focus was on proving the convergence of $\PP_{\pspace, \ndiscs, \nsamps, \imesh} (A) \to \paramP (A)$ for some $A\in \pborel$ and on estimating the error in $\PP_{\pspace, \ndiscs, \nsamps, h}(A)$.
There, as well as in \cite{BGE+15}, adjoint-based a posteriori estimates in the computed QoI are combined with a statistical analysis to both estimate and bound the error in $\PP_{\pspace, \ndiscs, \nsamps, \imesh} (A)$.
In \cite{BM17}, adjoints were used to compute both error and derivative estimates of $\qoi(\param^{(\iparam)})$ to improve the accuracy in $\PP_{\pspace, \ndiscs, \nsamps, \imesh} (A)$.
However, no work has to date fully explored the \emph{convergence rates} of Algorithm \ref{alg:inv_density}.
Furthermore, no work has yet to establish that these rates are independent of the choice of QoI map despite other studies establishing that the absolute error is very much affected by the geometric properties of the QoI maps \cite{BE13}.

In order to study convergence, we need to define a notion of distance on the space of probability measures on $\pspace$, which we denote by $\PPspace$.
% There are many choices available to us and we discuss several useful metrics on $\paramP$ in Section~\ref{sec:metrics}.
We will be using the Total Variation metric throughout this work, but for the time being, let $d$ represent any metric on $\PPspace$.

Then, we have by repeated application of the triangle inequality that
\begin{equation}
\label{eq:set-triangleineq}
d(\PP_{\pspace, \ndiscs, \nsamps, h}, \paramP) \leq
\underset{ \text{(E1)} }{\underbrace{d(\PP_{\pspace, \ndiscs, \nsamps, h},\PP_{\pspace, \ndiscs, \nsamps})}} +
\underset{ \text{(E2)} }{\underbrace{d(\PP_{\pspace, \ndiscs, \nsamps}, \PP_{\pspace, \ndiscs}) }}+
\underset{ \text{(E3)} }{\underbrace{d(\PP_{\pspace, \ndiscs}, \paramP) }}.
\end{equation}

The term (E1) describes the effect of the error in the numerically evaluated $\qoi_\iparam$ on the solution to the SIP.
The term (E2) describes the effect of finite sampling error in $\pspace$ on the solution to the SIP and (E3) describes the effect of discretization error of $\dataP$ on the solution to the SIP.
