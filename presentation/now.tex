%%%%%%%%%%%%%%%%%%%%%%%%%%%%%%%%%%%%%%%%%%%%%%%%%%%%%%%%%%%
%%%%%%%%%%%%%%%%%%%%%%%%%%%%%%%%%%%%%%%%%%%%%%%%%%%%%%%%%%%
\section{Parameter Identification and the MUD Point}
%%%%%%%%%%%%%%%%%%%%%%%%%%%%%%%%%%%%%%%%%%%%%%%%%%%%%%%%%%%
%%%%%%%%%%%%%%%%%%%%%%%%%%%%%%%%%%%%%%%%%%%%%%%%%%%%%%%%%%%


%%%%%%%%%%%%%%%%%%%%%%%%%%%%%%%%%%%%%%%%%%%%%%%%%%%%%%%%%%
\subsection{Theory of Regularization}
%%%%%%%%%%%%%%%%%%%%%%%%%%%%%%%%%%%%%%%%%%%%%%%%%%%%%%%%%%

\begin{frame}[t]
%\vskip 25pt
\centering
\begin{figure}
\centering

\emph{The one with the regularization equations.}

\begin{table}[htbp]
\centering

\begin{tabular}{|c|c|}
\hline
  & \\
  $\displaystyle \updated(\param) = \initial(\param) \frac{\observed(Q(\param))}{\predicted(Q(\param))}$
&
  $\displaystyle \pi_{\text{post}}(\param\,|\,d) = \frac{\pi_{\text{prior}}(\param)\pi_\text{like}(d\,|\,\param)}{\int_{\Lambda} \pi_\text{like}(d\, |\, \param)  \pi_{\text{prior}}(\param) d\pmeas}$
\\ 
\hline
  & \\
  Tikhonov & $T(\param):=\norm{Q(\param)-\observedMean}_{\observedCov^{-1}}^2 +
      \norm{\param-\initialMean}_{\initialCov^{-1}}^2$
  \\ & \\ \hline & \\
  Data-Consistent & $J(\param):=T(\param) - \norm{Q(\param)-Q(\initialMean)}_{\predictedCov^{-1}}^2$
\\
& \\
  \hline
\end{tabular}
\caption{The $\param$ which minimizes these functionals also maximizes the updated PDF (left) and the Bayesian posterior PDF (right).

$T(\param)$ is the typical functional often associated with Tikhonov regularization.

The $J(\param)$ has an additional term subtracted from $T(\param)$ coming from the predicted density that serves as ``unregularization'' in data--informed directions.}
  \label{tab:func_comparisons}
\end{table}


\end{figure}

\end{frame}

%%%%%%%%%%%%%%%%%%%%%%%%%%%%%%%%%%%%%%%%%%%%%%%%%%%%%%%%%%%
\subsection{Linear Example}
%%%%%%%%%%%%%%%%%%%%%%%%%%%%%%%%%%%%%%%%%%%%%%%%%%%%%%%%%%%

\begin{frame}[t]
%\vskip 25pt
\centering
\begin{figure}
\centering

The one with the contours.

\end{figure}

\end{frame}

%%%%%%%%%%%%%%%%%%%%%%%%%%%%%%%%%%%%%%%%%%%%%%%%%%%%%%%%%%%
\subsection{Impact of Information Content}
%%%%%%%%%%%%%%%%%%%%%%%%%%%%%%%%%%%%%%%%%%%%%%%%%%%%%%%%%%%

\begin{frame}[t]
%\vskip 25pt
\centering
\begin{figure}
\centering

The one where we show how rank and dimension impact our solutions.

\end{figure}

\end{frame}


%%%%%%%%%%%%%%%%%%%%%%%%%%%%%%%%%%%%%%%%%%%%%%%%%%%%%%%%%%%
\subsection{Data-Driven QoI Maps}
%%%%%%%%%%%%%%%%%%%%%%%%%%%%%%%%%%%%%%%%%%%%%%%%%%%%%%%%%%%

\begin{frame}[t]
%\vskip 25pt
\centering
\begin{figure}
\centering

The one where we show how to leverage this framework for general streams of data.

\end{figure}

\end{frame}


%%%%%%%%%%%%%%%%%%%%%%%%%%%%%%%%%%%%%%%%%%%%%%%%%%%%%%%%%%%
\subsection{Nonlinear Examples}
%%%%%%%%%%%%%%%%%%%%%%%%%%%%%%%%%%%%%%%%%%%%%%%%%%%%%%%%%%%

\begin{frame}[t]
%\vskip 25pt
\centering
\begin{figure}
\centering

The one where we show that our approach works even when some assumptions are violated.

\end{figure}

\end{frame}

