\subsection{Numerical Approximation and Analysis}\label{sec:sample-algorithm}

The numerical approximation of the updated density can be achieved by following Algorithm~\ref{alg:updated_density}, which involves approximating a push-forward distribution using a finite set of samples drawn from the initial density and mapped forward by $\qoi$.
Once an approximation of the predicted density is formed, it can be evaluated alongside the initial and observed density at any other point so long as evaluation through $\qoi$ is available.



\begin{algorithm}[hbtp]
\DontPrintSemicolon
Draw $\nsamps$ samples from the initial density to construct the set $\set{\param^{(\iparam)}}_{\iparam=1}^{\nsamps} \subset \pspace$.
	\For{$\iparam = 1, \hdots, \nsamps$}{
	Compute $\qoi_\iparam = \qoi(\param^{(\iparam)})$.\\
	}
	Approximate $\predicted\q$, the push-forward of $\initial$, by some method such as kernel density estimation.
  \For{$\iparam = 1, \hdots, \nsamps$}{
	Compute $\updated\lami = \initial\qi \frac{\observed\qi}{\predicted\qi}$.\\
	}

 \caption{Numerical Approximation of the Inverse Density using the Sample-Based Approach}
 \label{alg:updated_density}
\end{algorithm}

TK -
picture of pushforward given different number of samples, overview of KDE
