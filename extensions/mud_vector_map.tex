\section{Skewness and Vector-Valued Data-Constructed QoI Maps}
In our first example, we show that for the problem in \ref{subsec:pde-example}, a small change in the decision of how to aggregate the measurements collected on the response surface can induce a 2-dimensional map $\qoi_\text{2D}^\prime$ which is ``effectively'' one-dimensional; the two components provide highly correlated information.
Such a map is, by definition, more skewed than the one we presented with $\qoi_\text{2D}$.
Therefore, $\qoi_\text{2D}^\prime$ leads to MUD-point solutions that exhibit similar behavior to those obtained with $\qoi_\text{1D}$.
This decreased precision demonstrates that skewness\---although reviewed and analyzed in the previous chapter within the context of accuracy of set-valued solutions\---is a relevant measure of a map's utility for solving parameter identification problems.

The examples in Sections~\ref{subsec:ode-example} and \ref{subsec:pde-example} motivate the use of a data-constructed QoI in order to incorporate an arbitrary number of measurements in a system into a scalar-valued map.
These examples are chosen so that $\text{dim}({\Lambda}) = 1$ for simplicity and to establish a baseline for convergence results.
The linear examples in Section~\ref{sec:high-dim-linear-example} demonstrate that the DCI framework maintains the accuracy of least-squares while incorporating initial beliefs for higher-dimensional linear maps.
In those examples, we show that the ability to resolve a true parameter improves as the gap between input dimension and operator row-rank decreases.

The rank-deficiency of an operator is attributed to either the ill-conditioning of an operator $A:P\to P$, or when $P>D$ for a full-rank $A:P\to D$.
In scenarios where $S>P$ observations are available, we are motivated to leverage the form of Eq.~\eqref{eq:qoi_WME} to construct a vector-valued version of the QoI map incorporating subsets of $S$ for each component.
For example, a system for which spatial measurements are available over time may motivate constructing a scalar-valued QoI map using the WME functional for each spatial location.
If distinct observable quantities are available (e.g., perhaps with different physical units), then these may be collapsed into each component of the map.

The discussion of how to optimally construct such maps is beyond the scope of this work, and is highly problem specific, requiring nuances involving measurement sensitivities and combinatorial design-spaces.
However, we summarize that the extension of the equations presented in Section~\ref{sec:MUD_analysis} follows directly by constructing the resultant $1\times P$ matrices $A$ and scalar-valued $b$ for each component and then stacking them to form a $D\times P$ system, where we are motivated to minimize $P-D$.
The choice of how to form each of the (up to) $\dimP$ components will directly impact the precision of the parameter estimate.
