%%%%%%%%%%%%%%%%%%%%%%%%%%%%%%%%%%%%%%%%%%%%%%%%%%%%%%%%%%%%%%%%%%%%%%%%%%%%%%%%
% newcommands.tex

% itemize with the explaining text left justified afer the item
% Example:
%   \itembox{CNR} The Control net reduction
%   \itembox{A} Another item
% renders as
% * CNR   The Control net reduction
% * A     Another item
\newcommand{\itembox}[1]{\item {\makebox[0.80in]{#1 \hfill}}}

% Abbreviations 
\newcommand{\ie}{{\itshape i.e.}}
\newcommand{\eg}{{\itshape e.g.}}

% Bold symbols
\newcommand{\bs}[1]{\boldsymbol{#1}}

% Cardinality
\newcommand{\card}[1]{n\left(#1\right)} %{\left|#1\right|}

% Math Operators
\DeclareMathOperator*{\argmin}{arg\,min}
\DeclareMathOperator*{\argmax}{arg\,max}

% Standard Probability
\DeclareMathOperator{\E}{\mathbb{E}}
\DeclareMathOperator*{\P}{\mathbb{P}}

% Data-Consistent Inversion Framework


% Parameter Space

\DeclareMathOperator*{\param}{\lambda}
\DeclareMathOperator*{\pspace}{\Lambda}
\newcommand{\pmeas}{\mu_{\pspace}}
\newcommand{\pborel}{\mathcal{B}_{\pspace}}

\DeclareMathOperator*{\initialP}{\P_{\text{in}}}
\DeclareMathOperator*{\initial}{\pi_{\text{in}}}
\DeclareMathOperator*{\updatedP}{\P_{\text{up}}}
\DeclareMathOperator*{\updated}{\pi_{\text{up}}}

\DeclareMathOperator*{\data}{\boldsymbol{d}}
\DeclareMathOperator*{\dspace}{\mathcal{D}}
\newcommand{\dmeas}{\mu_{\dspace}}
\newcommand{\dborel}{\mathcal{B}_{\dspace}}

\DeclareMathOperator*{\q}{Q\left (\param \right )}
\DeclareMathOperator*{\M}{\mathcal{M}}

\DeclareMathOperator*{\obsP}{\P_{\text{ob}}}
\DeclareMathOperator*{\obs}{\pi_{\text{ob}}}
\DeclareMathOperator*{\predP}{\P_{\text{pr}}}
\DeclareMathOperator*{\pred}{\pi_{\text{pr}}}




%%%%%%%%%%%%%%%%%%%%%%%%%%%%%%%%%%%%%%%%%%%%%%%%%%%%%%%%%%%%%%%%%%%%%%%%%%%%%%%%

 