\chapter{\uppercase{Introduction}} \label{chapter:01}

\section{Preliminaries}

\
\section{Motivations}


\section{Software Contributions}
Everything in this thesis has been incorporated into open-source software.
The novel mathematical developments that have gone into the work herein are all reflected in various modules and sub-modules as part of the BET python package.
This software suite follows a number of industry best-practices for code-coverage and continuous integration, i.e. the code is well-tested.

A significant proportion of the effort involved in the writing of this thesis revolved around learning about the art and practice of modern (open-source) software development.
As new ideas sprang up, our research group found itself coding and re-coding the same methods that had yet to be incorporated into user-friednly, computationally efficient, and properly-parallelized libraries.
Over the years, the software fell behind the state-of-the-art in research, and previous maintainers of code had moved on from their academic positions.
The author spent most of 2019 bringing the software in-line with the latest developments in Data-Consistent Inversion.

\subsection{Architecture}
Having learned a lot about software reproducibility along the way, the author made the decision to treat this thesis as a software project in its own right.
Every example, figure, table, plot will be generated by a combination of Python and Bash scripts contained inside of an public Github repository (www.github.com/mathematicalmichael/thesis).
All the requisite LaTeX dependencies are contained in {\tt apt.txt}, and a Jupyterlab environment usable by Binder (which can compile the document and run every example) is configured in the {\tt binder/} directory of the repository.

Special care will be taken to ensure that every file herein is well-documented. When appropriate, functions and classes will be used in such a way that several examples can be generated from the same file.
The parameters required can be passed as optional arguments, and bash scripts containing the exact syntax to generate each figure will be included.

For example, to visually demonstrate the implicitly-defined sets of nearest-neighbors in two-dimensional unit domain, we rely on Voronoi-cell diagrams (figures).
One python file (\bashinline{images/voronoi_unit_domain.py}) contains the methods required to draw a figure.
Some plots require labels, and others do not, and at one point we want to demonstrate the impact of random sampling on the geometry of the induced computational equivalent of a $\sa$.
To accomodate these different plots, we utilize the argument-parsing package \pythoninline{argparse}, part of the Python standard library, to enable command-line positional and optional arguments \footnote{We equip each function with default values so that the syntax \bashinline{python example.py} without any additional arguments will work, but specific examples rely on properly-passed optional arguments.}.
Thus, we would include an associate (wrapper) file with a descriptive name, such as \bashinline{images/make_voronoi_diagrams.sh} that calls the relevant python file and passes arguments (such as {\tt num} for ``number of samples''):
\bashexternal{images/make_voronoi_diagrams.sh}.
