\chapter{\uppercase{Introduction}} \label{chapter:01}

\section{Preliminaries}
Talk about the collection of noisy data, the types of inferences one may be interested in making from them.
Roughly breaks down into 
\begin{itemize}
  \item Parameter Identification
  \item Distribution Estimation
\end{itemize}

Motivations break down by

\begin{itemize}
  \item Direct inference on something of interest
  \item Inference for the purpose of prediction
  \item Description of uncertainty around either of the aforementioned
\end{itemize}

\subsection{Motivations}\label{sec:motivations}
In some respects, the practice of writing software has diverged from the motivations of an academic researcher.
The latter seeks to generate new knowledge and may write a set of example scripts/programs to demonstrate some novel idea or method.
By contrast, the motivations of a software engineer are related to resiliency.
Not only must they ensure the code works as expected given a myriad of ways users may interact with it, but it is necessary to write the code in a manner compatible with maintaining it into the future.
Much of the work of writing ``good software'' is concerned with writing appropriate documentation to express the intended usage and logic underlying architectural decisions.
There are many ways to write a functioning program to demonstrate a proof-of-concept, but creating something that is \emph{user-friendly}, guaranteed to be free of mistakes, and scales across different computational environments/resources requires a completely different approach.

Decisions made early in the software design cycle have lasting impacts on future features and functionality.
Rigor is added to libraries through the writing of \emph{unit tests}, and use of \emph{continous integration} ensures that the download and installation process is predictable and reproducible. 
Code that only runs on the computer of the author is impractical since any thorough critique requires independent verification.
Without proper context and architecture, new ideas that are implemented in programs are unlikely to be adopted.

\subsection{Reproducibility}
To this end, this thesis is concerned not only with a demonstration of novel mathematical content\---showcasing new ways to make inferences from noisy data\---it also serves to document the process of ensuring that the work herein is \textbf{fully reproducible}.
In mathematics, reproducibility is ensured through the use of proofs, which motivate the original work presented here.
However, much of the focus is actually on the implications of implementation of the novel research into Data Consistent Inversion, studying the impact of using computers to perform the task of making conclusions based on data.
The implementation of mathematics on computers is done through software, and so we are concerned with ensuring the expected functionality of that software since as mathematicians, we care deeply about making sure things are rigorous. 

In short, we want to make sure that theory aligns with practice, and that both live up to high standards of intellectual scrutiny. 
Every computational result, illustrative figure, table, plot, etc. will have associated with it scripts that generate them, and will be included in the  repository for this document [TK - cite]. 
It is written in \LaTeX\, (which is itself a programming language), and presents its own software dependencies in addition to those required to run the scripts to generate the images and tables. 
To address this concern, the author will leverage \emph{Travis}, a continuous integration service [TK-cite], to ensure that all figures can be generated and the \LaTeX\, document compiles into a PDF.

The same care is taken to ensure the reproducibility of all results in this thesis as was put into ensuring that the mathematics was implemented correctly as software. 
Additionally, an \emph{image} that contains a fully pre-built Linux softare environment within which someone can compile the thesis and run the code will be available through the Docker Cloud registry [TK -cite]. 
The latter enables the ability to generate this thesis document in its entirety on any software platform that supports Docker (Windows, MacOS, Linux).

\section{Framework}\label{sec:framework}
We provide a summary of the notation, definitions, problem-formulation, and assumptions that reoccur throughout this work. 
For more details on the original sources and derivations,  we refer the interested reader to \cite{BES12, BE13, BET+14}. 

Let $u$ be the solution to a model $\M(u, \param) = 0$, perhaps of a physical system such as the amount of contaminant in a subsurface. 
Let $\param$ represent a parameter into such a model, e.g. the permeability of the medium in the subsurface through which a contaminant is spreading.
Such parameters are often uncertain and we begin the quantification of uncertainty by identifying the set of all physically plausible parameters denoted by $\pspace\subset\RR^\dimP$.
Since different choices of $\param \in \pspace$ often lead to different model solutions, we write $u\lam$ to make this dependence explicit.

However, we cannot in general observe the entire solution $u(\param)$.
Instead, we are often limited in our ability to observe data related to some quantities of interest (QoI), defined as functionals of $u\lam$ (e.g. the contaminant levels at a specific well at a particular time).
We let $\qoi$ denote the QoI map from the solution space of the model to the space of observable data. 
Then, given $\param \in \pspace$, we obtain $u\lam$ and compute $\qoi(u\lam)$ to get the QoI datum predicted by the model.
Clearly, the QoI map depends on $\param$ through the dependency of $u$ on $\param$, so we write $\qlam$ to simplify our notation.
We assume this map is at least piecewise-differentiable.	
The data space $\dspace \subset \RR^d$ is defined as the range of the QoI map $\qoi$, i.e. 
\[
\dspace = \qoi(\pspace).
\]
In other words, we use $\dspace$ to denote the space of all physically plausible data for the QoI that the model can predict.


Let $\pborel$ and $\dborel$ denote (the Borel) $\sigma$-algebras on $\pspace$ and $\dspace$, respectively.
The map $\qoi$ between measurable spaces $(\pspace, \pborel)$ and $(\dspace, \dborel)$ is immediately measurable by the smoothness assumption. 
Then, equipping $\pspace$ and $\dspace$ with (dominating) measures $\pmeas$ and $\dmeas$, respectively, is the final ingredient for defining probability density functions (pdfs) on the measure spaces $(\pspace, \pborel, \pmeas)$ and $(\dspace, \dborel, \dmeas)$.
In practice, $\pmeas$ and $\dmeas$ are often taken to be Lebesgue volume measures when $\pspace$ and $\dspace$ are finite-dimensional~\cite{BET+14, BJW18}.
In general, these measure allow for the description of probability measures as probability density functions defined by Radon-Nikodym derivatives.


\subsection{Problem Formulation and Solution}

We begin with defining the types of forward and inverse problems considered in this thesis.

\begin{defn}[Forward Problem]\label{defn:forward-problem}
  Given a probability measure $\PP_\pspace$ on $(\pspace, \pborel)$, and (at least piecewise-differentiable) QoI map $\qoi$, the \emph{forward problem} is to characterize a measure $\PP_\dspace$ on $(\dspace, \dborel)$ (recalling $\dspace = \qoi(\pspace)$) that satisfies
  \begin{equation}\label{eq:forward-problem}
    \PP_\dspace (E) = \PP_\pspace \left ( \qoi^{-1}(E) \right ), \; \forall \; E \in \dborel.
  \end{equation}
\end{defn}

\begin{defn}[Inverse Problem]\label{defn:inverse-problem}
  Given a probability measure $\observedP$ on $(\dspace, \dborel)$ that is absolutely continuous with respect to volume measure $\dmeas$, the \emph{inverse problem} is to determine a probability measure $\PP_\pspace$ on $(\pspace, \pborel)$, absolutely continuous with respect to $\pmeas$, satisfying

  \begin{equation}\label{eq:inverse-problem}
    \PP_\pspace (\qoi^{-1}(E)) = \int_{\qoi^{-1}(E)} \pp_\pspace \lam \, d\pmeas = \int_E \observed \q \, d\dmeas = \observedP(E), \; \forall \; E \in \mathcal{B}_\dspace.
  \end{equation} 

  \noindent Here,
   
  \begin{equation*}
    \pp_\pspace := \frac{d\PP_\pspace}{d\pmeas} \;\text{ and }\; \observed := \frac{d\observedP}{d\dmeas}
  \end{equation*}
  denote the Radon-Nikodym derivatives (i.e., pdfs) of $\updatedP$ and $\observedP$, respectively. 
  Any probability measure $\PP_\pspace$ satisfying \eqref{eq:inverse-problem} is referred to as a \emph{consistent solution} to the inverse problem, and \eqref{eq:inverse-problem} is referred to as the \emph{consistency condition}.
\end{defn}

In measure-theoretic terms, $\PP_\pspace$ is a pull-back measure of $\observedP$.
From the perspective of a forward problem, we seek $\PP_\pspace$ such that its \emph{push-forward measure is equivalent to} $\observedP$. 
In other words, the solution we seek to the inverse problem is constrained by a forward problem. 
Below, we formalize some of the vocabulary involved in the formulation and solution of the stochastic inverse problem.

\begin{defn}[Observed Distribution]\label{defn:observed}
  The density $\observed$ in \eqref{eq:inverse-problem} represents the uncertainty in QoI data and is referred to as the \emph{observed distribution} (or density), and is the Radon-Nikodym derivative of the \emph{observed measure} $\observedP$ with respect to the volume measure $\dmeas$.
\end{defn}

The map $\qoi$ impacts the structure of the update since the underlying data space $\dspace$ itself depends on $\qoi$, and both densities on $(\dspace, \dborel)$ are evaluated at $\qlam$.
In the event that the map $\qoi$ is a bijection, then the consistency condition \eqref{eq:inverse-problem} defines a unique measure $\PP$ on $\pspace$ given the specification of an observed density.
However, there are many applications of interest where $\qoi$ fails to be a bijection, either due to differences in the dimensions of the parameter and data spaces, nonlinearities inherent in the model itself, or both. 


The specific nuances of this relationship are discussed in \ref{chapter:02} and \ref{chapter:03} in greater detail.
Here, we simply note that the construction of \eqref{eq:update} requires only the forward-problem construction of $\predicted$, since $\initial$ and $\observed$ are given \emph{a priori}.
Additional properties are given in \ref{sec:properties} alongside the conditions for the existence and uniqueness of an update of the form given by \eqref{eq:inverse-problem}. 




%%%%%%%%% Section 2.2
\subsection{Properties and Assumptions of Consistent Update}\label{sec:properties}
The stochastic inverse problem is defined as finding a measure $\updatedP$ such that the push-forward of it matched $\observed$.
The following assumption guarantees a solution to the stochastic inverse problem. 
It implies that any event which is assigned a positive probability by the observations must also be assigned a positive probability of occurring by the prior beliefs. 

\begin{assumption}[Predictability Assumption]\label{as:pred}
The measure associated with $\observed$ is absolutely continuous with respect to the measure associated with $\observed$.
\end{assumption}

If this is unsatisfied, one source of information (the data) suggests certain events are probable while another source of information (the model and prior) have a priori ruled that almost surely these events should not occur. 
Therefore, either prior beliefs, the model under consideration, or the description of uncertainty encoded in $\predicted$ should be subjected to a critical reevaluation. 


The requirement given in Assumption~\ref{as:pred} is guaranteed if the following is satisfied:
\begin{equation}\label{eq:pred}
\exists \; C>0 \text{ such that } \observed (d) \leq C \predicted(d) \text{ for a.e. } d\in \dspace,
\end{equation}
where it is understood that $d = \qlam$ for some $\param \in \pspace$.
Now, assuming \eqref{eq:pred} holds, we restate the following theorem from \cite{BJW18}:

%%%

%Any probability measure that satisfies Equation~\ref{eq:inv} is considered a consistent solution to the stochastic inverse problem.
%In the event that the map $\qoi$ is a bijection, then the consistency condition (\ref{defn:consistency}) defines a unique measure $P$ on $\pspace$ given the specification of an observed density.
%However, there are many applications of interest where the quantitiy of interest map $\qoi$ fails to be a bijection, either due to differences in the dimensions of the parameter and data spaces, nonlinearities inherent in the model itself, or both. 
%To address this, we present the following summary of work done in \cite{BE13, BJW18} on two approaches for defining a unique solution if the properties of the map or spaces under consideration prohibit one from existing.

%%%
%\input{setval}


%%%%%%%%% Section 2.4
\subsection{Comparison to Other Methods}\label{sec:othermethods}
We establish two comparisons: first to statistical Bayesian inversion \cite{Walpole, Berger, Complete}, the second to a measure-theoretic approach studied in \cite{BET+14}. 


\subsubsection{Statistical Bayesian Approach}

%If the quantity of interest is a single measurement, then the likelihoods and observed densities are identical.
%However, the quantity of interest may not necessarily just be the uncertainty in the measurement data, as we will see in later discussions. 
%For completeness, we define the alternative solution to the SIP below and compare the two frameworks under a simple linear map.

The ``statistical Bayesian'' (the term we use herein for clarity) formulation gives a updated density as:
\begin{equation}\label{eq:sb_post}
    \hat{\updated}\lam := \initial\lam \frac{L_\dspace (d | \param)}{ C },
\end{equation}
where we use $\hat{\updated}$ to distinguish the posterior from the updated density in \eqref{eq:update}, $L_\dspace$ is the likelihood function as a function of the output and the denominator $C$ is a \emph{normalizing constant}, chosen so that the updated density integrates to one, namely
\[
C = \int_\pspace \initial\lam L_\dspace(d | \param) \, d\param.
\]

It is important to note that the statistical Bayesian framework poses a different question for which a different answer is sought. 
Specifically, the problem analyzed by the statistical Bayesian approach is to determine a single ``true'' parameter that explains all of the observed data \cite{Smith, Concrete, Complete}.
In this framework, there is a different notion of consistency, referring to certain asymptotic properties of $\updated$ in the limit of infinite data \cite{Barron, Silverman}.
This is in contrast to the consistent Bayesian framework, where we seek a pull-back measure: a description of the uncertainty set that explains the variation in the observations under a given description of error.
We note that there are problems that one can formulate where the observed density corresponds directly to a normalized likelihood function familiar to Bayesian statisticians, as we demonstrate below. 

One difference that is immediately obvious between the two solutions is the use of normalizing constant $C$ in $\hat{\updated}$ not present in $\updated$, as discussed in Corollary~\ref{cor:int} above.
To illustrate the two approaches, we explore the impact of this difference in the example below, taken from \cite{BJW18}.

\begin{ex}
Consider
\begin{equation}
u(\param) = \param^p
\end{equation}
for $p$ chosen as either 1 or 5. 
For both approaches, the prior is given by $\initial \sim U[-1,1]$. 
In the consistent Bayesian approach, $\observed \sim \mathcal{N}(0.25,0.1^2)$.
In the statistical Bayesian framework, we take $d=0.25$ and assume an additive error noise model with distribution $\mathcal{N}(0,0.1^2)$ so that the likelihood, $L_\dspace(d | \param)$, matches the observed density as a function of $\param$.

When $p=1$, we have $\predicted = \frac{1}{2} = C$.
Since $\observed\q = L_\dspace(d|\param)$, the posterior and update agree on $\pspace$ (we show the push-forwards in the right plot of Figure~\ref{fig:comparison}). 
When $p=5$, the non-linearity of the model causes the push-forward of $\initial$ to be non-constant, so the two approaches yield differing solutions, as seen in the left plot of Figure~\ref{fig:comparison}).
The push-forward of the posterior associated with the statistical Bayesian framework is influenced by the push-forward of the prior in a way that the push-forward of the updated density avoids.

\begin{figure}\label{fig:comparison}

\begin{minipage}{.45\textwidth}
		\includegraphics[width=\linewidth]{./images/comparison1}
\end{minipage}
\begin{minipage}{.45\textwidth}
		\includegraphics[width=\linewidth]{./images/comparison5}
\end{minipage}
\caption{In each plot, the black dotted line represents the prior while the solid line represents the push-forward of the prior. The push-forwards of the posterior and consistent updated density are shown as green and blue dotted lines, respectively. The observed density/likelihood is shown as a solid red line. (Left): $p = 1$, the push-forwards of the solutions are identical because a linear map results in a constant push-forward of a uniform prior. (Right): $p = 5$, the non-linearity of the map causes the solutions to be different and thus the push-forwards to also be different.}
\end{figure}

We can see in Figure~\ref{fig:comparison} that the two approaches pose different questions and thus yield different results. 
The consistent Bayesian framework seeks to recreate the observed density, which it does for both $p=1$ and $p=5$, but the statistical approach is a weighted sum of both the observed and the push-forward of the prior.
\end{ex}

\section{Software Contributions}\label[sec:ch01-software]
As discussed in \ref{sec:motivations}, the entirety of the mathematical content herein has been incorporated into freely available open-source software.
The novel mathematical developments that have gone into the work herein are all reflected in various modules and sub-modules as part of the BET python package.
This software suite follows a number of industry best-practices for code-coverage and continuous integration, i.e. the code is well-tested.

A significant proportion of the effort involved in the writing of this thesis revolved around learning about the art and practice of modern (open-source) software development.
As new ideas sprang up, our research group found itself coding and re-coding the same methods that had yet to be incorporated into user-friednly, computationally efficient, and properly-parallelized libraries.
Over the years, the software fell behind the state-of-the-art in research, and previous maintainers of code had moved on from their academic positions.
The author spent most of 2019 bringing the software in-line with the latest developments in Data-Consistent Inversion.

\subsection{Architecture}\label{sec:architecture}
Having learned a lot about software reproducibility along the way, the author made the decision to treat this thesis as a software project in its own right.
Every example, figure, table, plot will be generated by a combination of Python and Bash scripts contained inside of an public Github repository (www.github.com/mathematicalmichael/thesis).
All the requisite LaTeX dependencies are contained in {\tt apt.txt}, and a Jupyterlab environment usable by Binder (which can compile the document and run every example) is configured in the {\tt binder/} directory of the repository.

Special care will be taken to ensure that every file herein is well-documented.
When appropriate, functions and classes will be used in such a way that several examples can be generated from the same file.
The parameters required can be passed as optional arguments, and bash scripts containing the exact syntax to generate each figure will be included.

For example, to visually demonstrate the implicitly-defined sets of nearest-neighbors in two-dimensional unit domain, we rely on Voronoi-cell diagrams (figures).
One python file (\bashinline{images/voronoi_unit_domain.py}) contains the methods required to draw a figure.
Some plots require labels, and others do not, and at one point we want to demonstrate the impact of random sampling on the geometry of the induced computational equivalent of a $\sa$.
To accomodate these different plots, we utilize the argument-parsing package \pythoninline{argparse}, part of the Python standard library, to enable command-line positional and optional arguments \footnote{We equip each function with default values so that the syntax \bashinline{python example.py} without any additional arguments will work, but specific examples rely on properly-passed optional arguments.}.
Thus, we would include an associate (wrapper) file with a descriptive name, such as \bashinline{images/make_voronoi_diagrams.sh} that calls the relevant python file and passes arguments (such as {\tt num} for ``number of samples''):
\bashexternal{images/make_voronoi_diagrams.sh}.
