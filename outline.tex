\section{Outline}

\begin{description}[leftmargin=!, labelwidth=0.7in]
\item[Chapter 1] Introduction

\item[Chapter 2] Background on Data Consistent Inversion:
\begin{description}[leftmargin=!, labelwidth=0.7in]
	\item[Section 2.1] Notation, Terminology, and Assumptions
	\item[Section 2.2] Set-Based Inversion for Measures
	\item[Section 2.3] Sample-Based Inversion for Densities
	\item[Section 2.4] Software Contributions (description of your updating of BET to Python 3 that you are doing for NSF goes here along with including the newer approach with densities that you are adding to BET, discuss testing, installation, etc. at a high level)
	\item[Section 2.5] Illustrative Examples
\end{description}

\item[Chapter 3] Impact of Output Quantities on Accuracy
\begin{description}[leftmargin=!, labelwidth=0.7in]
\item[Section 3.1] Skewness and Information Content (this is a review section)
\item[Section 3.2] Skewness and Accuracy of Set-Based Inversion (this is a summary of your MS work updated for TV metric)
\item[Section 3.3] Skewness and Accuracy of Sample-Based Inversion (newer work that you were doing but hadn't written up yet, focus on linear problems and how KDEs deal with skewness on the data space)
\item[Section 3.4] Software contributions (adding the module to BET that computes the TV metric that is updated from your MS work, also discuss testing, and simple examples of usage that are disconnected from skewness, mostly high level)
\item[Section 3.5] Numerical results and analysis
\end{description}

\item[Chapter 4] Data-driven maps and Consistent Inversion
\begin{description}[leftmargin=!, labelwidth=0.7in]
\item[Section 4.1] A Generalized Stochastic Map Framework (material from the paper I am finishing up -- hopefully in the next two weeks for first draft -- goes here to set the stage)
\item[Section 4.2] Data-driven maps (also material from the paper including sensitivity analysis as the number of data points $M$ increases)
\item[Section 4.3] Software contributions (adding a module in BET to transform time series into QoI and do the data-consistent inversion)
\item[Section 4.4] Numerical Results and Analysis
\end{description}

\item[Chapter 5] Other Research from NSF project yet to be done/defined well enough to sketch out this chapter - perhaps functional assimilation discussion? Maybe no chapter 5 on this? We will see.
 
\item[Chapter 6] Summary, Conclusions and Future Research Directions

\item[Bibliography]

\end{description}