\section{Towards a Reproducible Thesis}\label{sec:reproducibility}

% \subsection{Motivations}\label{sec:motivations}
In some respects, the practice of writing software has diverged from the motivations of an academic researcher.
The latter seeks to generate new knowledge and may write a set of example scripts/programs to demonstrate some novel idea or method.
By contrast, the motivations of a software engineer are related to resiliency.
Not only must they ensure the code works as expected given a myriad of ways users may interact with it, but it is necessary to write the code in a manner compatible with maintaining it into the future.
Much of the work of writing ``good software'' is concerned with writing appropriate documentation to express the intended usage and logic underlying architectural decisions.
Without proper context and an understandable architecture, new ideas that are implemented in programs are unlikely to be adopted.
There are many ways to write a functioning program to demonstrate a proof-of-concept, but creating something that is \emph{user-friendly}, and scales to different computational environments/resources, requires an entirely different approach.

Decisions made early in the software design cycle have lasting impacts on future features and functionality.
Rigor is added to libraries through the writing of \emph{unit tests}, and eventually \emph{functional tests}, which validate individual components and entire workflows, respectively.
The practice of \emph{continuous integration} ensures that the download and installation process is predictable and reproducible by running the requisite steps (and tests) in an ephemeral environment as an independent verification that programs execute as expected.
Code that only runs on the one's computer is impractical, since any thorough review of the results requires validation by an independent third party.
Having continuous integration (and deployment for packaging the software), tests, and documentation, allows a repository of code to be functionally and practically accessible to the larger research community.

This thesis is concerned not only with a demonstration of novel mathematical content\---showcasing new ways to make inferences from noisy data in a novel Data-Consistent framework\---it also serves to set a precedent for guaranteeing that the results presented are \textbf{fully reproducible}.
In mathematics, reproducibility is ensured through the use of proofs, which motivate the original work presented here.
However, as the title of this thesis suggests, much of the work involves computational implementation of the novel research into Data Consistent Inversion, studying the impact of using computers to perform the task of making conclusions based on data.
Mathematics is implemented on computers through software.
We are therefore concerned with verifying and validating the expected functionality of that software, which aligns with our training as mathematicians; we care deeply about making sure things are rigorous.

In short, we want to make sure that theory aligns with practice, and that both live up to high standards of intellectual scrutiny.
Every computational result, illustrative figure, table, plot, etc. presented in this thesis is associated with the scripts that generate them, and are included in the publicly-available GitHub repository.
With a minimal set of instructions, everything (including this document itself), can be reproduced.
The specific tools and frameworks through which the reproduction of these results can be accomplished change over time.
The GitHub repository for this dissertation will provide a number of pathways for generating the results, including ones that do not require any installation on local or remote resources.

\FloatBarrier
