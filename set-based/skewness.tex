% \section{Skewness}\label{sec:skewness}
In \cite{BGE+15}, the concept of skewness in a QoI map $\qoi$ is introduced, quantified, and related to the accuracy in solving the stochastic inverse problem with a finite number of samples.
In effect, skewness is a geometric property that describes how the right angles in generalized rectangles belonging to $\dborel$ are transformed by $\qoi^{-1}$.
An a priori analysis demonstrated that the number of samples from a {\em regular uniform grid} in $\pspace$ required to approximate the $\pmeas$-measure of $\qoi^{-1}(E)$ to a desired level of accuracy is proportional to the skewness of $\qoi$ raised to the ($d-1$) power where $d$ is the dimension of $\dspace$.
This is a version of the so-called curse-of-dimension for the set-based approach.

Skewness is explored further in \cite{Walsh} in the context of optimal experimental design.
There, an additional geometric property of $\qoi$ related to the \emph{precision} in the solution of the associated stochastic inverse problem is introduced and quantified.
For completeness, we define skewness below and refer the interested reader to \cite{BGE+15, BPW17, Walsh} for more details.

\begin{defn}
For any QoI map $\qoi$, $\param \in \pspace$, and a specified row vector $\bf{j}_k$ of the Jacobian $J_{\param, Q}$, we define
\begin{equation}
S_\qoi(J_{\param,Q}, \bf{j}_k) := \frac{\abs{\bf{j}_k} }{\abs{\bf{j}_k^\perp}}.
\label{eq:skewness}
\end{equation}

We define the \textbf{local skewness} of a map $\qoi$ at a point $\pspace$ as
\begin{equation}
S_\qoi(\param) := \max_{1\leq k \leq d} S_\qoi(J_{\param,Q}, \bf{j}_k).
\label{eq:localskewness}
\end{equation}
\end{defn}

\begin{defn}
The \textbf{average} \emph{(or \textbf{expected})} \textbf{skewness} is defined as
\begin{equation}
\overline{S_Q} := \frac{1}{\mu_{\pspace}(\pspace)} \int_\pspace S_Q (\param) \, d\mu_{\pspace}
\label{eq:avgskew}
\end{equation}
\end{defn}

In \cite{BPW17}, it is shown that $S_\qoi(\param)$ is efficiently computed using a singular value decomposition (SVD) of the Jacobian $J_{\param,\qoi}$, i.e., we randomly sample $J_{\param,\qoi}$ and compute the SVDs.
In general, we approximate $\overline{S_\qoi}$ with Monte-Carlo approximations.
